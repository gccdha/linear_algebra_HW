\documentclass[12pt,oneside]{amsart}
\thispagestyle{empty}
\usepackage{lmodern}
\usepackage[
%scale=0.75,
margin=1in,
]{geometry}
\usepackage{versions}
\newif\ifsol\solfalse

%\soltrue % comment to hide solution

\ifsol \newenvironment{solution}{\par\noindent\textbf{Solution:\\}}{\hfill\qed\\} 
\else \excludeversion{solution} \fi
\ifsol \newenvironment{grading}{\par\noindent\textbf{Grading:}}{\hfill\\}
\else \excludeversion{grading} \fi

%\newcommand{\ba}{\mathbf{a}}
%\newcommand{\bb}{\mathbf{b}}
%\newcommand{\be}{\mathbf{e}}
\newcommand{\bo}{\mathbf{0}}
%\newcommand{\bp}{\mathbf{p}}
%\newcommand{\bq}{\mathbf{q}}
\newcommand{\bu}{u}
\newcommand{\bv}{v}
\newcommand{\bw}{w}
\newcommand{\bx}{x}
\newcommand{\by}{y}
\newcommand{\Q}{\mathbb{Q}}
\newcommand{\R}{\mathbb{R}}
\newcommand{\Z}{\mathbb{Z}}

\usepackage{mdframed}
\newcommand
\sol[1]{
\medskip
\begin{mdframed}
\textbf{Ans:\\} #1
\end{mdframed}
\medskip
}

\DeclareMathOperator{\Span}{Span}
\DeclareMathOperator{\Nul}{Nul}
\DeclareMathOperator{\Col}{Col}
\DeclareMathOperator{\Row}{Row}
\DeclareMathOperator{\id}{id}


\newcommand{\vt}[2]{\left[\begin{matrix} #1 \\ #2 \end{matrix}\right]}
\newcommand{\vd}[3]{\left[\begin{smallmatrix} #1 \\ #2 \\ #3 \end{smallmatrix}\right]}



\begin{document}
\begin{center}
{\Large\textbf{Math 2135 - Assignment 11}}\\
\medskip
Due November 15, 2021 \\
Maxwell Rodgers
\end{center}
\bigskip
\thispagestyle{empty}


\begin{enumerate}
\item
 Compute the determinants by row reduction to echelon form:

\[ A = \left[\begin{matrix}
3 & 3 & -3 \\
3 & 4 & -4 \\
2 & -3 & -5 \end{matrix}\right] \hspace{1cm}
B = \left[\begin{matrix}
1 & 3 & 2 & -4 \\
0 & 1 & 2 & -5 \\
2 & 7 & 6 & -3 \\
-3 & -10 & -7 & 2
\end{matrix}\right] \]

\sol{
  \begin{enumerate}
    \item
      $ |A| = \left|\begin{matrix}
        3 & 0 & 0 \\
        0 & 1 & -1 \\
        0 & 0 & -8 \end{matrix}
        \right| = (3)(1)(-8) = -24$
    \item
      $
        |B| = \left|\begin{matrix}
        1 & 3 & 2 & -4 \\
        0 & 1 & 2 & 5 \\
        0 & 0 & 1 & 5 \\
        0 & 0 & 0 & -10
        \end{matrix}\right|
        =  (-10)(1)(1)(1) = -10
      $
  \end{enumerate}
}

\item
 Let $A=\begin{bmatrix} a & b \\ c & d \end{bmatrix}$ and $B=\begin{bmatrix} u & v \\ w & x \end{bmatrix}$.
 Show  
\[ \det(AB) = \det(A)\det(B). \]

\sol{
  $\det(A)\det(B)
  =(ad-bc)(ux-vw)
  =adux-advw-bcux+bcvw
  =(acuv)+adux+bcvw+(bdwx)-(acuv)-advw-bcux-(bdwx)
  =(au+bw)(cv+dx)-(av+bx)(cu+dw)
  =\left|\begin{matrix}au+bw & av+bx \\ cu+dw & cv+dx \end{matrix}\right|
  =\det(AB)$
}

\item Let $A\in\R^{n\times n}$. Are the following true or false? Explain why:
\begin{enumerate}
\item    
 If two rows or columns of $A$ are identical, then $\det A = 0$.   
\item
 For $c\in\R$, $\det (cA) = c\det A$.
\item
 If $A$ is invertible, then $\det A^{-1} = \frac{1}{\det A}$. 
\item
 $A$ is invertible iff $0$ is not an eigenvalue of $A$.
\end{enumerate}

\sol{
  \begin{enumerate}
    \item TRUE\\
      This means that the row reduced form of the matrix will have a zero row which means det A = 0.
    \item FALSE\\
      Since the determinant multiplies elements of the matrix, $\det(cA)$ will be more like $c^n \det A$
    \item TRUE\\
      Since we just showed that $\det(AB)=\det(A)\det(B)$, $\det(AA^{-1})=\det(I)=1=\det(A)\det(A^{-1})$ meaning that the determinant of A and the inverse of A must be multiplicative inverses of each other.
    \item TRUE\\
      If A has an eigenvalue of 0 this means that the nullspace of A is non-trivial (bc. there exists $x\ne0$ s.t. $Ax=0$) and therfore the columns of A are not linearly independant and A doesn't have an inverse.
  \end{enumerate}
}


 
\item  
 Eigenvalues, -vectors and -spaces can be be defined for linear maps just as for matrices.

 Let $h\colon V\to W$ be a linear map for vector spaces $V,W$ over $F$. Show that the eigenspace for $\lambda\in F$,
\[ E_{h,\lambda} := \{ x\in V  \ :\ h(x) = \lambda x \}, \]
 is a subspace of $V$. 

 \sol{
  We know that $E_{h,\lambda}$ must be a subspace of $V$ because
  \begin{enumerate}
    \item[1)] all the elements of $E_{h,\lambda}$ are in $V$ as well
    \item[2)] since $E_{h,\lambda}$ is an eigenspace it must be closed under addition and scalar multiplication.
    \item[3)] it includes the zero element of v ($\lambda(x)$ for $\lambda=0$ or equivilant)
  \end{enumerate}

 }
  
\item
 Give all eigenvalues and bases for eigenspaces of the following matrices.
 Do you need the characteristic polynomials? \\
  $A = \left[\begin{matrix} -3 & 1 \\ 0 & -3 \end{matrix}\right]$ \hfill 
  $B = \left[\begin{matrix} 2 & 0 & 0  \\  1 & 0 & 0 \\ -1 & 0 & 3 \end{matrix}\right]$
  \sol{
   The Eigenvalue of $A$ is -3 and the Eigenvalues for $B$ are 0,2, and 3.\\
   This means that the eigenspace of $A$ is spanned by the vector $\vt10$ 
   (from row reduced augmented matrix $\left[\begin{matrix} 0 & 1 & 0 \\ 0 & 0 & 0 \end{matrix}\right] $)\\
   and the eigenspaces of $B$ are spanned by the vectors $\vd010$, $\vd{1}{\frac12}{1}$, and$\vd001$
   \\ (from the nullspaces of 0:$\left[\begin{matrix} 1 & 0 & 0 \\ 0 & 0 & 3 \\ 0 & 0 & 0\end{matrix}\right],2:\left[\begin{matrix} -1 & 0 & 1 \\ 0 & -2 & 1 \\ 0 & 0 & 0\end{matrix}\right],3:\left[\begin{matrix} -1 & 0 & 0 \\ 0 & -3 & 0 \\ 0 & 0 & 0 \end{matrix}\right]$)\\
   We don't need to use the characteristic polynomial because these matrices are both triangular.
  }

\item
 Give the characteristic polynomial, all eigenvalues and bases for eigenspaces for
 $C = \left[\begin{matrix} 1 & 2 \\ 3 & 1 \end{matrix}\right]$.

 \sol{
  $(1-\lambda)^2-6=0)$ is the characteristic polynomial so the eigenvalues are $1\pm\sqrt6$\\
  We can find the eigenvectors by reducing the matrix $\left[\begin{matrix} \mp\sqrt6 & 2 \\ 3 & \mp\sqrt6 \end{matrix}\right]$\\
  $\left[\begin{matrix} \mp\sqrt6 & 2 \\ 0 & 0 \end{matrix}\right]$ which gives the eigenvectors $\vt{\frac{2}{\mp\sqrt6}}{1}=\vt{-\frac{\sqrt6}{3}}{1},\vt{\frac{\sqrt6}{3}}{1}$
 }

\item
 Compute eigenvalues and eigenvectors for
 $D = \left[\begin{matrix} -1 & 4 & 1 \\ 6 & 9 & 2 \\ 0 & 0 & -3 \end{matrix}\right]$.

 \sol{
  Polynomial: $(-3-\lambda)((-1-\lambda)(9-\lambda)-24)=-(\lambda+3)(\lambda+3)(\lambda-11)=0$\\
  This means the eigenvalues are -3 and 11.
  Taking the matrices $\left[\begin{matrix} 1 & 2 & 0 \\ 0 & 0 & 1 \\ 0 & 0 & 0 \end{matrix}\right]$ and $\left[\begin{matrix} 3 & -1 & 0 \\ 0 & 0 & 1 \\ 0 & 0 & 0\end{matrix}\right]$
    which corespond to the eigenvectors $\vd210$ and $\vd{\frac13}{1}{0}$
 }

\item
 Are the matrices $A,B,C,D$ in (5), (6), (7) diagonalizable? How?
 \sol{
  Both B and C are diagonalizable but A and D are not because they do not have as many eigenspaces as they have rows/columns.\\
  The diagonalized matrices of B and C would just be the eigenvalues of those matrices put in the diagonal of an n by n matrix with all the other entries set to 0.
 }

\end{enumerate}
\end{document}


\item Let $A$ be an $n\times n$-matrix. Are the following true or false? Explain why:
\begin{enumerate}
\item
 If $A$ has $n$ eigenvectors, then $A$ is diagonalizable.
\item
 If a $4\times 4$-matrix $A$ has two eigenvalues with eigenspaces of dimension $3$ and $1$, respectively,
 then $A$ is diagonalizable.
\item
 $A$ is diagonalizable iff $A$ has $n$ eigenvalues (counting multiplicities).
\item
 If $\R^n$ has a basis of eigenvectors of $A$, then $A$ is diagonalizable.
\end{enumerate}


\item
 Let $A\in\R^{n\times n}$ with $n$ eigenvalues $\lambda_1,\dots,\lambda_n$ (repeated according
 to their multiplicities). Show that 
\[ \det A = \lambda_1\cdot\lambda_2\cdots \lambda_n \]
 Hint: Consider the characteristic polynomial $\det(A-\lambda I) = (\lambda_1-\lambda)\cdots(\lambda_n-\lambda)$.

\end{enumerate}






\end{document}




