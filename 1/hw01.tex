\documentclass[12pt,a4paper]{exam}
\thispagestyle{empty}
%\usepackage{versions}
%\newif\ifsol\solfalse


\usepackage{amsmath}
\usepackage{amssymb}
\usepackage[lmargin=71pt, tmargin=1.2in]{geometry}  %For centering solution bo


\newcommand{\R}{\mathbb{R}}
\newcommand{\Nul}{\mathrm{Nul}}




\begin{document}
\begin{center}
\begin{Large}
{\bf Math 2135 - Assignment 1}
\end{Large}

\bigskip
Due September 6, 2024\\Completed September 5th by Maxwell Rodgers
\end{center}

\bigskip
\printanswers

 Solve all systems of linear equations by row reduction (Gaussian elimination) and give the solutions in
 parametric vector form.

\renewcommand{\solutiontitle}{\noindent\textbf{Ans:}\enspace}   %Replace "Ans:" with starting keyword in solution box
    
\begin{enumerate}
\item
 Do the following 4 planes intersect in a point? Which?
\begin{align*}
 x+5y+3z & = 16 \\
 2x+10y+8z & = 34 \\
 4x+20y+15z & = 67 \\
 x+6y+5z & = 21
\end{align*}

\begin{solution}
    \begin{itemize}
        \item Turn equations into augmented matrix:\\
            \[ \begin{matrix} I \\ II \\ III  \\ IV \end{matrix} \left[ \begin{array}{cccc}
            1 & 5 & 3 & 16 \\
            2 & 10 & 8 & 34 \\
            4 & 20 & 15 & 67 \\
            1 & 6 & 5 & 21
            \end{array} \right] \]
        \item Row reduce and re-arrange:\\
        (subtract 2I,4I and I from II, III and IV respectively, multiply II by $\frac{1}{2}$, subtract 3II from I, then move IV between I and II)
            \[ \begin{matrix} I \\ II \\ III  \\ IV \end{matrix} \left[ \begin{array}{cccc}
            1 & 5 & 3 & 16 \\
            0 & 1 & 2 & 5  \\
            0 & 0 & 1 & 1 \\
            0 & 0 & 0 & 0 
            \end{array} \right] \]
        \item Solve for $x$, $y$ and $z$:\\
            $z=1$ from III. \\
            $y=5-2=3$ from II and $z$. \\
            $x=16-3-15=-2$ from I $y$ and $z$. \\
            So the point at which all the given planes intersect at is $(-2,3,1)$.
    \end{itemize}
\end{solution}

\item 
 Add an equation of a line to the equation
\[ 2x+3y = 4 \]
 such that the resulting system has (a) no solution, (b) exactly one solution,
 (c) infinitely many solutions.


 \begin{solution}
    \begin{enumerate}
        \item No Solution: $-2x-3y=0$\\
            \[
            \begin{array}{rcl}
            2x + 3y &=& 4 \\
            -2x - 3y &=& 0 \\
            \hline
             0 &=& 4
            \end{array}
            \]
            The contradiction shows that there are no solutions
        \item One Solution: $-2x-2y=-3$\\
            \[
            \begin{array}{rcl}
            2x + 3y &=& 4 \\
            -2x - 2y &=& -3 \\
            \hline
             y&=& 1
            \end{array}
            \]
            This implies that $x=\frac{1}{2}$, leading to the single solution $(\frac{1}{2},1)$
        \item Infinite Solutions: $4x+6y=8$\\
            \[
            \begin{array}{rcl}
            2x + 3y &=& 4 \\
            -\frac{1}{2}(4x + 6y &=& 8) \\
            \hline
             0&=&0
            \end{array}
            \]
            The result of $n=n$ where in this case $n=0$ implies the lines are the same and thus have infinite solutions.
    \end{enumerate}
 \end{solution}

\item
 Solve the system of linear equations with augmented matrix
\[\begin{matrix} I \\ II \\ III \end{matrix} \left[ \begin{array}{cccc}
 0 & 0 & 2 & 4 \\
 2 & -4 & 1 & 0 \\
 -3 & 6 & 2 & 7
 \end{array} \right] \]

\begin{solution}
    \begin{itemize}
        \item Row Reduce:\\ do $\frac{3}{2}$II + III, then do $\frac{2}{7}$III and $\frac{1}{2}$I. Next, do III-I and I-II followed by $\frac{1}{2}$I. Finally, swap I and II to get the following matrix.
            \[\begin{matrix} I \\ II \\ III \end{matrix} \left[ \begin{array}{cccc}
            1 & -2 & 0 & -1 \\
            0 & 0 & 1 & 2  \\
            0 & 0 & 0 & 0
            \end{array} \right] \]
        \item Solve Equations:\\
            $x_3=2$ and $x_1=2x_2-1$. Taking $x_2=t$ as the parameter, we get the following equation:
            $$x=\left(\begin{matrix}
                -1\\0\\2
            \end{matrix} \right) + 
            t\left(\begin{matrix}
                2\\1\\0
            \end{matrix} \right)$$
    \end{itemize}
\end{solution}

\item
 Solve the system of linear equations with augmented matrix
\[ \begin{matrix}
    I\\II\\III
\end{matrix}\left[ \begin{array}{ccccc}
 2 & 2 & 1 & 8 & 2 \\
 2 & 0 & 0 & 8 & 2 \\
 2 & 6 & 3 & 8 & 1
 \end{array} \right] \]

\begin{solution}
    \begin{itemize}
        \item Row Reduce:\\
            Start by subtracting II from I and III. Then subtract 3I from III. at this point I stopped the row reduction leaving me with the following matrix.
            \[ \left[ \begin{array}{ccccc}
            0 & 2 & 1 & 0 & 0 \\
            2 & 0 & 0 & 8 & 2 \\
            0 & 0 & 0 & 0 & -1
            \end{array} \right] \]
        \item Solve Equations:\\
            I stopped the row reduction because based on the final row of the matrix, we have the equation $0=-1$ indicating that this system has no solutions.
    \end{itemize}
\end{solution}

\item
  Let $A \in\R^{3\times 2}$. %be a $3 \times 2$ matrix.
  Explain why $Ax=b$ cannot
  have a solution $x$ for all $b \in \R^3$.
  \begin{solution}
      Since $A$ represents three equations that have only two variables, the most lenient case is all the equations being the same (since none would be limited by the others). This would lead to the solution being an equation with two variables. This cannot represent the entirety of $\R^3$ as having two variables is essentially having two degrees of freedom. This means that the most that $A$ can represent is $\R^2$, and if it were graphed in 3D it would at most be a plane.
  \end{solution}


\item \cite[cf. Section 1.5, Ex 17]{La-LA}
Let
\[
A=\left[\begin{matrix} 2 & 2 & 4 \\ -4 & -4 & -8 \\ 0 & -3 & -3
\end{matrix}\right], \quad
b=\left[\begin{matrix} 8 \\ -16 \\ 12 \end{matrix}\right], \quad
\mathbf{0}=\left[\begin{matrix} 0 \\ 0 \\ 0 \end{matrix}\right].
\]
Solve $Ax=b$ and  $Ax=\mathbf{0}$. Express both solution sets in parametric vector form.
Give a geometric description of the solution sets.

\begin{solution}
    \begin{itemize}
        \item $Ax=b$:\\
            \[ \begin{matrix}
                I\\II\\III
            \end{matrix} \left[ \begin{array}{cccc}
            2 & 2 & 4 & 8 \\
            -4 & -4 & -8 & -16 \\
            0 & -3 & -3 & 12
            \end{array} \right] \]
        \item Row Reduce:\\
            Start by doing $\frac{1}{2}$I, $\frac{1}{4}$II and $\frac{-1}{3}$III. Then add I to II and subtract III from I. Finally, swap II and III to get the following matrix.
            \[ \left[ \begin{array}{cccc}
            1 & 0 & 1 & 8 \\
            0 & 1 & 1 & -4 \\
            0 & 0 & 0 & 0
            \end{array} \right] \]
        \item Solve Equations:\\
            The two equations are $x+y=8$ and $z+y=-4$. Using $y=t$ we get the following equation:
            $$x=\left(\begin{matrix}
                8\\0\\-4
            \end{matrix} \right) + 
            t\left(\begin{matrix}
                -1\\1\\-1
            \end{matrix} \right)$$

            
        \item $Ax=0$:\\
            \[ \begin{matrix}
                I\\II\\III
            \end{matrix} \left[ \begin{array}{cccc}
            2 & 2 & 4 & 0 \\
            -4 & -4 & -8 & 0 \\
            0 & -3 & -3 & 0
            \end{array} \right] \]
        \item Row Reduce:\\
            Start by doing $\frac{1}{2}$I, $\frac{1}{4}$II and $\frac{-1}{3}$III. Then add I to II and subtract III from I. Finally, swap II and III to get the following matrix.
            \[ \left[ \begin{array}{cccc}
            1 & 0 & 1 & 0 \\
            0 & 1 & 1 & 0 \\
            0 & 0 & 0 & 0 
            \end{array} \right] \]
        \item Solve Equations:\\
            The two equations are $x_1+x_2=0$ and $x_3+x_2=0$. Using $x_2=t$ we get the following equation:
            $$x=\left(\begin{matrix}
                0\\0\\0
            \end{matrix} \right) + 
            t\left(\begin{matrix}
                -1\\1\\-1
            \end{matrix} \right)$$
    \end{itemize}
\end{solution}



\item \cite[cf. Section 1.5, Ex 11]{La-LA}
Let
\[
A=\left[\begin{matrix} 1 & -4 & -2 & 0 & 3 & -5 \\ 0 & 0 & 1 & 0 & 0 & -1 \\ 0 & 0 & 0 & 0 & 1 & -4
\end{matrix}\right], \quad
b=\left[\begin{matrix} 1 \\ 1 \\ 1 \end{matrix}\right], \quad
\mathbf{0}=\left[\begin{matrix} 0 \\ 0 \\ 0 \end{matrix}\right].
\]
Solve the equations $Ax=b$ and  $Ax=\mathbf{0}$. Express both solution sets in parametric vector form.

\begin{solution}
    \begin{itemize}
        \item $Ax=b$:\\
            \[ \begin{matrix}
                I\\II\\III
            \end{matrix} \left[ \begin{array}{ccccccc}
            1 & -4 & -2 & 0 & 3 & -5 & 1\\
            0 & 0 & 1 & 0 & 0 & -1 & 1\\ 
            0 & 0 & 0 & 0 & 1 & -4 & 1
            \end{array} \right] \]
        \item Row Reduce:\\
            Start by adding 2II to I then subtract 3III from I.
            \[ \left[ \begin{array}{ccccccc}
            1 & -4 & 0 & 0 & 0 & 5 & 0\\
            0 & 0 & 1 & 0 & 0 & -1 & 1\\ 
            0 & 0 & 0 & 0 & 1 & -4 & 1
            \end{array} \right] \]
        \item Solve Equations:\\
            The three equations are $x_1=4x_2-5x_6$, $x_3=x_6+1$, and $x_5=4x_6+1$. Using $x_2=t$, $x_6=u$, and $x_4=v$ we get the following equation:
            $$x=\left(\begin{matrix}
                0\\0\\1\\0\\1\\0
            \end{matrix} \right) + 
            t\left(\begin{matrix}
                4\\1\\0\\0\\0\\0
            \end{matrix} \right) + 
            u\left(\begin{matrix}
                -5\\0\\1\\0\\4\\1
            \end{matrix} \right) + 
            v\left(\begin{matrix}
                0\\0\\0\\1\\0\\0
            \end{matrix} \right)$$

            
        \item $Ax=b$:\\
            \[ \begin{matrix}
                I\\II\\III
            \end{matrix} \left[ \begin{array}{ccccccc}
            1 & -4 & -2 & 0 & 3 & -5 & 0\\
            0 & 0 & 1 & 0 & 0 & -1 & 0\\ 
            0 & 0 & 0 & 0 & 1 & -4 & 0
            \end{array} \right] \]
        \item Row Reduce:\\
            Start by adding 2II to I then subtract 3III from I.
            \[ \left[ \begin{array}{ccccccc}
            1 & -4 & 0 & 0 & 0 & 5 & 0\\
            0 & 0 & 1 & 0 & 0 & -1 & 0\\ 
            0 & 0 & 0 & 0 & 1 & -4 & 0
            \end{array} \right] \]
        \item Solve Equations:\\
            The three equations are $x_1=4x_2-5x_6$, $x_3=x_6$, and $x_5=4x_6$. Using $x_2=t$, $x_6=u$, and $x_4=v$ we get the following equation:
            $$x=\left(\begin{matrix}
                0\\0\\0\\0\\0\\0
            \end{matrix} \right) + 
            t\left(\begin{matrix}
                4\\1\\0\\0\\0\\0
            \end{matrix} \right) + 
            u\left(\begin{matrix}
                -5\\0\\1\\0\\4\\1
            \end{matrix} \right) + 
            v\left(\begin{matrix}
                0\\0\\0\\1\\0\\0
            \end{matrix} \right)$$

    \end{itemize}
\end{solution}

\item
 Are the following true or false? Explain your answers.

\begin{enumerate}
\item
 Any system of linear equations with strictly less equations than variables has infinitely many solutions.

\item
 Different sequences of elementary row reductions can transform one matrix to distinct matrices in
 row echelon form.

\item
 A consistent system has exactly one solution.

\end{enumerate}

\begin{solution}
    \begin{enumerate}
        \item True\\
            Each equation only allows you to solve for one variable, potentially in terms of other variables. If there are $n$ equations and $n+1$ variables, there will always be one variable that cannot be solved for.
        \item True\\
            It is possible to end up with distinct but equivalent matrices. i.e. some rows might be multiplied by a factor in one matrix compared to another matrix, still representing the same system of equations. If we add the restriction of the pivot being 1, then it is no longer possible, as this does not allow any row to be multiplied by any factor other than the one that leads the pivot to equal 1.
        \item False\\
            All that a consistent system is is a system that is not unsolvable. This means that a consistent system could have infinitely many or finitely many  solutions.
    \end{enumerate}
\end{solution}

\end{enumerate}


\begin{thebibliography}{1}
\bibitem{La-LA}
 David C. Lay, Steven R. Lay, and Judi J. McDonald.
\newblock Linear Algebra and Its Applications.
\newblock Addison-Wesley, 5th edition, 2015.
\end{thebibliography}

\end{document}

