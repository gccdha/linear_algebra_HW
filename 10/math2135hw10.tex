\documentclass[12pt,oneside]{amsart}
\thispagestyle{empty}
\usepackage{lmodern}
\usepackage[
%scale=0.75,
margin=1in,
]{geometry}
\usepackage{versions}
\newif\ifsol\solfalse

%\soltrue % comment to hide solution

\ifsol \newenvironment{solution}{\medskip\par\noindent\textbf{Solution:\\}}{\hfill\qed\medskip} 
\else \excludeversion{solution} \fi
\ifsol \newenvironment{grading}{\par\noindent\textbf{Grading:}}{\hfill\\}
\else \excludeversion{grading} \fi

\newcommand{\id}{P}
%\newcommand{\bb}{\mathbf{b}}
%\newcommand{\be}{\mathbf{e}}
\newcommand{\bo}{\mathbf{0}}
%\newcommand{\bp}{\mathbf{p}}
%\newcommand{\bq}{\mathbf{q}}
\newcommand{\bu}{u}
\newcommand{\bv}{v}
\newcommand{\bw}{w}
\newcommand{\bx}{\mathbf{x}}
\newcommand{\by}{y}
\newcommand{\Q}{\mathbb{Q}}
\newcommand{\R}{\mathbb{R}}
\newcommand{\Z}{\mathbb{Z}}

\usepackage{mdframed}
\newcommand
\sol[1]{
\medskip
\begin{mdframed}
\textbf{Ans:\\} #1
\end{mdframed}
\medskip
}

\DeclareMathOperator{\Span}{Span}
\DeclareMathOperator{\Nul}{Nul}
\DeclareMathOperator{\Col}{Col}
\DeclareMathOperator{\Row}{Row}


\newcommand{\vt}[2]{\left[\begin{matrix} #1 \\ #2 \end{matrix}\right]}
\newcommand{\vd}[3]{\left[\begin{smallmatrix} #1 \\ #2 \\ #3 \end{smallmatrix}\right]}


\begin{document}
\begin{center}
{\Large\textbf{Math 2135 - Assignment 10}}\\
\medskip
Due November 8, 2024 \\
\end{center}
\bigskip
\thispagestyle{empty}


{\bf Problems 1-5 are review material for the second midterm on November 6.
Solve them before Wednesday!}

\bigskip

\begin{enumerate}
\item
  Let $V, W$ be vector spaces over $\R$ with zero vectors $0_V , 0_W$, respectively.
  Let $f\colon V\to W$ be linear. Show
\begin{enumerate}
\item $f (0_V ) = 0_W$ ,
\item the kernel $\ker f := \{v\in V \ :\ f(v) = 0_W \}$ of $f$ is a subspace of $V$.
\end{enumerate}

\sol{
  \begin{enumerate}
    \item
      the zero vector in V multiplied by x (that represents the linear map) will always end up being zero (since there is only multiplication) in the output.
    \item
      We know from the previous part that $\ker f$ is a subset of $V$.\\
      The kernel is closed under addition because $f(v)=0$ and $0+0=0$\\
      The kernel is closed under multiplication because $f(v)=0$ and $0n=0$ for all $n\in\R$
  \end{enumerate}
}

\item
 Let $T\colon P_2 \to \R, p\mapsto p(3)$, be the map that evaluates a polynomial $p$ at $x=3$.
\begin{enumerate}
\item Show that $T$ is linear.
\item Determine the kernel of $T$, that is, $\ker T = \{ p \in P_2 \ :\ T(p) = 0 \}$,
 and the image of $T$, that is, $T(P_2)$.
\item Is $T$ injective, surjective, bijective?
\end{enumerate}  

\sol{
  \begin{enumerate}
    \item
      Property 1: For $a,b\in P_2$ it is true that $a(3)+b(3)=(a+b)(3)$ (since the polynomials don't change each other in the proccess of addition)\\
      Property 2: For $a \in P_2$ and $n\in\R$ it is true that $n\cdot a(3) = (n\cdot a)(3)$ (multiplying the result or all of the terms yeilds the same outcome)
    \item
      The kernel is where the polynomials equal zero when evaluated at 3 so the kernel is all polynomials of degree 2 with the root $(x-3)$.\\
      The image is just $\R$ because any real number can be output by a polynomial of degree 2. 
    \item
      It is not injective because many polnomials output the same value for 3, but it is surjective since the range is the same as the codomain.
    
  \end{enumerate}
}

\item
 Let $B = (b_1,b_2)$ with $b_1 = \vd{-5}{11}{5}, b_2 = \vd{3}{-1}{4}$ and
  $C = (\vd{1}{1}{3},\vd{2}{-2}{1})$ be bases of a subspace $H$ of $\R^3$.
\begin{enumerate}
\item
 Compute the coordinates $[b_1]_C$ and $[b_2]_C$.
\item 
 What is the change of coordinate matrix $P_{C\leftarrow B}$?
\item  
 What is the change of coordinate matrix $P_{B\leftarrow C}$? 
\end{enumerate}

\sol{
 \begin{enumerate}
   \item
     Augmented matrix: $\left[\begin{matrix}1 & 2 & -5 & 3\\ 1 & -2 & 11 & -1 \\ 3 & 1 & 5 & 4 \end{matrix}\right]\to\left[\begin{matrix}1 & 0 & 3 & 1 \\ 0 & 1 & -4 & 1 \\ 0 & 0 & 0 & 0 \end{matrix}\right]$\\
     So $[b_1]_C=\vt{3}{-4},[b_2]_C=\vt{1}{1}$
   \item
     Using the vectors we just found, $P_{C\leftarrow B}=\left[\begin{matrix}3&1\\-4&1\end{matrix}\right]$
   \item
     Now we just take the inverse of $P_{C\leftarrow B}$ to get $P_{B \leftarrow C} = \frac{1}{7}\cdot \left[\begin{matrix}1&-1\\4&3\end{matrix}\right]$
 \end{enumerate}
}

\item
  Let $C = (1+t,t+t^2,1+t^2)$ be a basis for $P_2$.  Compute the coordinates
 $[p]_C$ for $p = 2+t^2$.

 \sol{
   we can represent this problem as the augmented matrix\\
   $\left[\begin{matrix} 1&0&1&2\\1&1&0&0\\0&1&1&1 \end{matrix}\right] \to \left[\begin{matrix} 1&0&0&\frac12\\0&1&0&-\frac12\\0&0&1&\frac32 \end{matrix}\right]$ so the answer is $\vd{\frac12}{-\frac12}{\frac32}$
 }

\item
\begin{enumerate}
\item
 Show that $A\in\R^{n\times n}$ is invertible iff rank $A = n$.
\item
 If $A$ is a $3\times 4$-matrix, what is the largest possible rank of $A$?
 What is the smallest possible dimension of Nul $A$? 
\item 
 If the nullspace of a $4\times 6$-matrix $B$ has dimension $3$, what is the dimension of the row space
 of $B$?
 \end{enumerate}

 \sol{
  \begin{enumerate}
    \item
      $A$ is invertible iff its columns are linearly independant. $A$'s columns are linearly independant iff rank $ A=n$
    \item
      The largest possible rank of A is 3 (3 rows) and the smallest possible dimension of the nullspace is 1(1 more column than row).
    \item
      The dimension of the rowspace(/columnspace) and the dimension of the nullspace must add up to the number of columns so Row A has dimension 3.
  \end{enumerate}
 }


\item
 Compute the determinant of the matrices by cofactor expansion. Pick a row or column that yields the least
 amount of computation:
\[ A = \left[\begin{matrix}
0 & 1 & -3 \\
5 & 4 & -4 \\
0 & -3 & -4 \end{matrix}\right] \hspace{1cm}
 B = \left[\begin{matrix}
1 & 0 & -3 & 0 \\
3 & 1 & 5 & 1 \\
2 & 0 & 0 & 0 \\
7 & 1 & -2 & 5 \\
\end{matrix}\right].
\]

\sol{
  \begin{enumerate}
    \item
      Using row 2 we just have to do 1 calculation: $\det A=-5\left|\begin{matrix} 1 & -3 \\ -3 & -4 \end{matrix}\right|=(-5)(-13)=65$
    \item
      Using row 3 in B we have to do 1 $3\times3$ matrix, and if we do the first row in that matrix we only have to do one again: $\det B=2\left|\begin{matrix} 0 & -3 & 0 \\ 1 & 5 & 1 \\ 1 & -2 & 5 \end{matrix}\right|=(2)(3)\left|\begin{matrix}1 & 1 \\ 1 & 5\end{matrix}\right|=(2)(3)(4)=24$
  \end{enumerate}
}

\item
 {\bf Rule of Sarrus for the determinant of $3\times 3$-matrices.}
 Let
\[ A = \left[\begin{matrix}
 a_{11} & a_{12} & a_{13} \\
 a_{21} & a_{22} & a_{23} \\
 a_{31} & a_{32} & a_{33} \end{matrix}\right] \]
 Prove that
\[ \det A =  a_{11}a_{22}a_{33}+a_{12}a_{23}a_{31}+a_{13}a_{21}a_{32} - a_{13}a_{22}a_{31}-a_{11}a_{23}a_{32}-a_{12}a_{21}a_{33} \]
 Hint: Expand $\det A$ across the first row.

 \sol{
  $$\left|\begin{matrix}
 a_{11} & a_{12} & a_{13} \\
 a_{21} & a_{22} & a_{23} \\
 a_{31} & a_{32} & a_{33} \end{matrix}\right| = a_{11} \left|\begin{matrix} a_{22} & a_{23} \\ a_{32} & a{33} \end{matrix}\right|-a_{12} \left|\begin{matrix} a_{21} & a_{23} \\ a_{31} & a{33} \end{matrix}\right|+ a_{13} \left|\begin{matrix} a_{21} & a_{22} \\ a_{31} & a{32} \end{matrix}\right|=$$ $$a_{11}a_{22}a_{33} - a_{11}a_{23}a_{32} - a_{12}a_{21}a_{33} + a_{12}a_{23}a_{31} + a_{13}a_{21}a_{32} - a_{13}a_{22}a_{31}=$$$$a_{11}a_{22}a_{33}+a_{12}a_{23}a_{31}+a_{13}a_{21}a_{32} - a_{13}a_{22}a_{31}-a_{11}a_{23}a_{32}-a_{12}a_{21}a_{33}$$
 }


\item
 Consider $A = \left[\begin{matrix} a & b \\ c & d \end{matrix}\right]$.
\begin{enumerate}
\item
 How does switching the rows effect the determinant? Compare $\det A$ and
 $\det \left[\begin{matrix} c & d \\ a & b \end{matrix}\right]$
\item
 How does multiplying one row by a scalar effect the determinant?
 Compare $\det A$ and $\det \left[\begin{matrix} ra & rb \\ c & d \end{matrix}\right]$. 
\item
 How does adding a multiple of one row to the other row effect the determinant?
 Compare $\det A$ and $\det \left[\begin{matrix} a & b \\ c+ra & d+rb \end{matrix}\right]$. 
\end{enumerate}

\sol{
  \begin{enumerate}
    \item
      It multiplies the determinant by -1: $\det A=ad-bc$ and $\det \left[\begin{matrix} c & d \\ a & b \end{matrix}\right]=bc-da=-\det A$
    \item
      It multiplies the determinant by r: $\det \left[\begin{matrix} ra & rb \\ c & d \end{matrix}\right]=rad-rbc=r(ad-bc)=r\det A$ 
    \item
      $\det \left[\begin{matrix} a & b \\ c+ra & d+rb \end{matrix}\right]=ad+rab-bc-rab=ad-bc=\det A$
  \end{enumerate}
}

\end{enumerate}
\end{document}



