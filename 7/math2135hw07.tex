\documentclass[12pt,oneside]{amsart}
\thispagestyle{empty}
\usepackage{lmodern}
\usepackage[
%scale=0.75,
margin=1in,
]{geometry}
\usepackage{versions}
\newif\ifsol\solfalse

%\soltrue % comment to hide solution

\ifsol \newenvironment{solution}{\par\noindent\textbf{Solution:\\}}{\hfill\qed\\} 
\else \excludeversion{solution} \fi
\ifsol \newenvironment{grading}{\par\noindent\textbf{Grading:}}{\hfill\\}
\else \excludeversion{grading} \fi

\newcommand{\ba}{\mathbf{a}}
\newcommand{\bb}{\mathbf{b}}
\newcommand{\be}{\mathbf{e}}
\newcommand{\bo}{\mathbf{0}}
\newcommand{\bp}{\mathbf{p}}
\newcommand{\bq}{\mathbf{q}}
\newcommand{\bu}{\mathbf{u}}
\newcommand{\bv}{\mathbf{v}}
\newcommand{\bw}{\mathbf{w}}
\newcommand{\bx}{\mathbf{x}}
\newcommand{\by}{\mathbf{y}}
\newcommand{\Z}{\mathbb{Z}}
\newcommand{\R}{\mathbb{R}}

\usepackage{mdframed}
\newcommand\sol[1]{
\medskip
\begin{mdframed}
\textbf{Ans:\\} #1
\end{mdframed}
\medskip
}

\DeclareMathOperator{\Span}{Span}
\DeclareMathOperator{\Nul}{Nul}
\DeclareMathOperator{\Col}{Col}


\newcommand{\vt}[2]{\left[\begin{matrix} #1 \\ #2 \end{matrix}\right]}
\newcommand{\vd}[3]{\left[\begin{matrix} #1 \\ #2 \\ #3 \end{matrix}\right]}
\newcommand{\vf}[4]{\left[\begin{matrix} #1 \\ #2 \\ #3 \\ #4 \end{matrix}\right]}


\begin{document}
%\maketitle
\begin{center}
{\Large\textbf{Math 2135 - Assignment 7}}\\
\bigskip
Due October 18, 2024 \\
\end{center}
\bigskip
\thispagestyle{empty}


\begin{enumerate}
 \item
 Explain why the following are not subspaces of $\R^2$.
 Give explicit counter examples for subspace properties that are not satisfied. 
 \begin{enumerate}
 \item $U = \{ \vt{x}{y} \mid x,y\in\R, x\geq 0 \}$
 \item $V = \Z^2$ \quad ($\Z$ denotes the set of all integers)
 \item $W = \{ \vt{x}{y} \mid x,y\in\R, |x|=|y| \}$
 \end{enumerate}
 
 \sol{
  \begin{enumerate}
    \item This is not a subspace because it is not closed under scaling. For example, 
      $-1\cdot \vt{1}{2} = \vt{-1}{-2} \notin U$ because $-1 < 0$.
    \item This is not a subspace because it is not closed under scaling. For example,
      $0.2\cdot \vt{1}{2} = \vt{0.2}{0.4} \notin V$ because $0.2 \notin \Z$
    \item This is not a subspace because it is not closed under addition. For example,
      $\vt{1}{1} + \vt{-1}{1} = \vt{0}{2} \notin W$ because $0\neq-2$
  \end{enumerate}
 }

\item
 Which of the following are subspaces of the vector space $\R^\R = \{ f \colon \R \to \R \}$ of all functions from
 $\R$ to $\R$? Check all subspace properties or give one that is not satisfied.
\begin{enumerate}
\item $\{ f \colon \R \to \R \mid f(0) = 1 \}$
\item $\{ f \colon \R \to \R \mid f(3) = 0 \}$
\item $\{ f \colon \R \to \R \mid f \text{ is continuous} \}$
\end{enumerate}

\sol{
  \begin{enumerate} 
    \item
      Not a subspace\\ This is not a subspace because it is not closed under addition. If we add two functions where $f(0)=1$, we will get a function where $f(0)=2$.
    \item
      Subspace\\
      \begin{enumerate}
        \item
          Addition\\$f(3)+f(3)=0+0=0$ So it is closed under addition.
        \item
          Scaling\\$af(3)=a\cdot0=0$ So it is closed under scaling.
        \item
          Zero vector\\Contains the zero vector $f(x)=0$.
      \end{enumerate}
    \item
      Subspace\\
      \begin{enumerate}
        \item
          Addition\\Two continuous functions added cannot create a discontinuous function (closed under addition).
        \item
          Scaling\\Scaling a continuous function creates a new continuous function (closed under scaling).
        \item
          The zero vector $f(x)=0$ is continuous.
      \end{enumerate}

  \end{enumerate}
}

\item Let $\bv_1,\ldots,\bv_n$ be vectors in a vector space $V$.
Show that $U := \Span\{\bv_1,\ldots,\bv_n\}$ is a subspace of $V$.
\sol{
  U must be a subspace of V, because it can be shown that it has all the required properties:
  \begin{itemize}
    \item[1.] By definition $U \subseteq V$ because $\bv_1,\ldots,\bv_n\in V$
    \item[2.] Since span is the linear combination of all the vectors, and that includes multiplying them all by zero, U contains the zero element.
    \item[3\&4.] Since span is the linear combination of all the vectors, this includes all possible scalings and additions of those vectors, making their span closed under addition and scaling.
  \end{itemize}
}


\item
Let $A\in\R^{m \times n}$.
Prove that $\Nul(A)$ is a subspace of $\R^n$.

\sol{
  The nullspace of A must be a subspace of $\R^n$ because it can be represented as the span of some number of n dimensional vectors representing the set of homogenous linear equations defined by $Ax=0$.
  As was found in the previous question, the fact that this is a span means that it is a subspace.
}


\item Explain whether the following are true or false (give counter examples if possible):
\begin{enumerate}
\item  
 Every vector space is a subspace of itself.
\item
 Each plane in $\R^3$ is a subspace.
\item
 Let $U$ be a subspace of a vector space $V$. Any linear combination of vectors of $U$ is also in $V$.
\item
 Let $v_1,\dots, v_n$ be in a vector space $V$. Then $\Span(v_1,\dots, v_n)$ is the smallest subspace of $V$
 containing $v_1,\dots, v_n$. 
\end{enumerate}

\sol{
  \begin{enumerate}
    \item \textbf{True}\\
      Every vector space is by definition closed under addition and multiplication, has its own zero vector and is a(n impropper) subset of itself.
    \item \textbf{False}\\
      Some planes don't have the zero vector in them.Ex: $x=5$
    \item \textbf{True}\\
      The elements in U are closed over linear combination, and U's elements are by definition a subset of V's, so all the linear combinations of elements in U are in V.
    \item \textbf{True}\\
      Since the subspace must be closed under addition and scaling, any subspace containing the vectors $v_1,\dots, v_n$ must be at least as large as $\Span(v_1,\dots, v_n)$.
  \end{enumerate}
}


\item Are the vectors $\bv_0 = 1$, $\bv_1 = t$, $\bv_2 =t^2$ in the vector space $\R^\R := \{ f \colon \R \to \R \}$ linearly independent?

  \sol{
    Yes, they are linearly independant because in order to go from $\bv_0$ to $\bv_1$ to $\bv_2$ you would have to multiply by $t$ in each step. The multiplication allowed in vector spaces is only by constant scalars, not variables, and there is no other way to generate these elements from one another.
  }

\item
 Which of the following are bases of $\R^3$? Why or why not?
\[ A = ( \vd{1}{2}{0}, \vd{2}{3}{4} ), B = ( \vd{1}{2}{0}, \vd{2}{3}{4}, \vd{0}{-1}{4} ), C = (\vd{1}{2}{0}, \vd{2}{3}{4}, \vd{0}{1}{1} ) \]

\sol{
  \begin{enumerate}
    \item
      Not a basis\\ The basis for $\R^3$ must have 3 vectors in it because it is 3 dimensional.
    \item
      Not a basis\\ $-2\cdot\vd{1}{2}{0}+\vd{2}{3}{4}=\vd{0}{-1}{4}$ so they are not linearly independant.
    \item
      Basis\\ By row reducing the matrix $\left[\begin{matrix} 1 & 2 & 0 \\ 2 & 3 & 1\\ 0 & 4 & 1 \end{matrix}\right]$ We get $\left[\begin{array}{rrr}
1 & 2 & 0 \\
0 & -1& 1 \\
0 & 0 & 5
\end{array}\right]$. We can see from this that since there are no free variables and a pivot in every row, the vectors are independent and thus are a basis of $\R^3$ 
  \end{enumerate}
}


\item Give a basis for Nul$(A)$ and a basis for Col$(A)$ for
\[ A = \left[ \begin{matrix}
 0 & 2 & 0 & 3 \\
 1 & -4 & -1 & 0 \\
 -2 & 6 & 2 & -3
 \end{matrix} \right] \]


\sol{
  First, we put A in reduced echelon form:
  $$\left[\begin{array}{rrrr}
1 & 0 & -1 & 6 \\
0 & 1 & 0 & \frac{3}{2} \\
0 & 0 & 0 & 0
\end{array}\right]$$
  \begin{itemize}
    \item \textbf{Nul(A):} Using the echelon form of A, we can get a set of homogenous equations:
      $$x_1=x_3-6x_4, x_2=-\frac{3}{2}x_4$$
      From this we get the solution to the homogenous equation: $x=s\vf{1}{0}{1}{0}+t\vf{-6}{-\frac{3}{2}}{0}{1}$ where $s=x_3$ and $t=x_4$.
      These two vectors are the basis for the nullspace.
    \item \textbf{Col(A):}
      We see that there are two pivot columns in the echelon form of the matrix. We take what these columns were in the original matrix as the basis of the column-space. That is, $\vd{0}{1}{-2}$ and $\vd{2}{-4}{6}$ is the basis. 
  \end{itemize}
}

\end{enumerate}
\end{document}


