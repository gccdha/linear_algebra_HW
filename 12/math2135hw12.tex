\documentclass[12pt,oneside]{amsart}
\thispagestyle{empty}
\usepackage{lmodern}
\usepackage[
%scale=0.75,
margin=0.95in,
]{geometry}
\usepackage{versions}
\newif\ifsol\solfalse

%\soltrue % comment to hide solution

\ifsol \newenvironment{solution}{\par\noindent\textbf{Solution:\\}}{\hfill\qed\\} 
\else \excludeversion{solution} \fi

%\newcommand{\ba}{\mathbf{a}}
%\newcommand{\bb}{\mathbf{b}}
%\newcommand{\be}{\mathbf{e}}
\newcommand{\bo}{\mathbf{0}}
%\newcommand{\bp}{\mathbf{p}}
%\newcommand{\bq}{\mathbf{q}}
\newcommand{\bu}{u}
\newcommand{\bv}{v}
\newcommand{\bw}{w}
\newcommand{\bx}{x}
\newcommand{\by}{y}
\newcommand{\Q}{\mathbb{Q}}
\newcommand{\R}{\mathbb{R}}
\newcommand{\Z}{\mathbb{Z}}

\usepackage{mdframed}
\newcommand
\sol[1]{
\medskip
\begin{mdframed}
\textbf{Ans:\\} #1
\end{mdframed}
\medskip
}

\DeclareMathOperator{\Span}{Span}
\DeclareMathOperator{\Nul}{Nul}
\DeclareMathOperator{\Col}{Col}
\DeclareMathOperator{\Row}{Row}
\DeclareMathOperator{\proj}{proj}


\newcommand{\vt}[2]{\left[\begin{matrix} #1 \\ #2 \end{matrix}\right]}
\newcommand{\vd}[3]{\left[\begin{smallmatrix} #1 \\ #2 \\ #3 \end{smallmatrix}\right]}



\begin{document}
\begin{center}
{\Large\textbf{Math 2135 - Assignment 12}}\\
\medskip
Due Nov 22, 2024 \\
Maxwell Rodgers
\end{center}
\bigskip
\thispagestyle{empty}


\begin{enumerate}

\item
 Are the matrices $A,B,C,D$ in (5), (6), (7) of assignment 11 diagonalizable? How?

 \sol{
  Both B and C are diagonalizable but A and D are not because they do not have as many eigenspaces as they have rows/columns.\\
  The diagonalized matrices of B and C would just be the eigenvalues of those matrices put in the diagonal of ann by n matrix with all the other entries set to 0.
 }

\item Let $A$ be an $n\times n$-matrix. Are the following true or false? Explain why:
\begin{enumerate}
\item
 If $A$ has $n$ eigenvectors, then $A$ is diagonalizable.
\item
 If a $4\times 4$-matrix $A$ has two eigenvalues with eigenspaces of dimension $3$ and $1$, respectively,
 then $A$ is diagonalizable.
\item
 $A$ is diagonalizable iff $A$ has $n$ eigenvalues (counting multiplicities).
\item
 If $\R^n$ has a basis of eigenvectors of $A$, then $A$ is diagonalizable.
\item 
 Every triangular matrix is diagonalizable. 
\end{enumerate}

\sol{
  \begin{enumerate}
    \item ?\\
    \item TRUE\\
    \item FALSE\\
    \item TRUE\\
    \item FALSE\\
  \end{enumerate}
}

\item
 Let $A$ be the standard matrix for the reflection $t$ of $\R^2$ on some line $g$ throught the origin.
 What are the eigenvalues, eigenvectors and eigenspaces of $A$? Can $A$ be diagonalized? \\
 Hint: Consider what a reflection does to specific vectors. 

\item
 As the previous problem for a rotation $r$ of $\R^2$ by an angle $\varphi$ around the origin. \\
 Hint: Consider $\varphi = 0, \pi$ separately.
 
\item
Consider a population of owls feeding on a population of squirrels.
In month $k$, let $o_k$ denote the number of owls and $s_k$ the number of squirrels.
Assume that the populations change every month as follows:
\begin{align*}
o_{k+1} &= 0.3 o_k + 0.4 s_k \\
s_{k+1} &= -0.4 o_k + 1.3 s_k
\end{align*}
That is, if there would be no squirrels to hunt, only $30\%$ of the owls would survive to the next month;
if there were no owls that hunted squirrels, then the squirrel population would grow by factor $1.3$
every month.

Let $x_k=\vt{o_k}{s_k}$.
Express the population change from $x_k$ to $x_{k+1}$ using a matrix $A$. Diagonalize $A$.

\sol{
  The matrix is $A=\left[\begin{matrix} 0.3 & 0.4 \\ -0.4 & 1.3 \end{matrix}\right]$ (from the cofficients of the equations)
  This means that the eigenvalues are represented by the equation 
  $(0.3-\lambda)(1.3-\lambda)+0.16 = \lambda^2-1.6\lambda+0.64-0.09=(\lambda-0.8)^2-0.09=0$ so $\lambda=0.8\pm0.3$
  This means that the diagonalized matrix is $D=\left[\begin{matrix} 1.1 & 0 \\ 0 & 0.5 \end{matrix}\right]$
  To get the $P$ matrix we take the $D-I\lambda$ matrices and find their nullspaces: 
  $0.5:\left[\begin{matrix}-0.2& 0.4\\-0.4&0.8\end{matrix}\right]\rightarrow\vt{2}{1}$ and
  $1.1:\left[\begin{matrix}-0.8& 0.4\\-0.4&0.2\end{matrix}\right]\rightarrow\vt{1}{2}$ meaning the P
  matrix is $\left[\begin{matrix} 1 & 2\\2&1\end{matrix}\right]$
}

\item Continue the previous problem: 
Let the starting population be $x_1 = \vt{o_1}{s_1} = \vt{20}{100}$.
\begin{enumerate}
%\item Express the population change using a matrix $A$. Diagonalize $A$.
\item Give an explicit formula for the populations in month $k+1$.
\item
Are the populations growing or decreasing over time? By which factor?
\item What is ratio of owls to squirrels after 12 months? After 24 months? Can you explain why?
\end{enumerate}
 
\sol{
  \begin{enumerate}
    \item
      First we find the starting parameters in terms of the eigenvectors which is simply
      $\vt{o_{k+1}}{s_{k+1}}=\left[\begin{matrix} 1 & 2 \\ 2& 1 \end{matrix}\right]\left[\begin{matrix} 1.1 & 0 \\ 0 & 0.5 \end{matrix}\right]^n\left[\begin{matrix} -\frac13 & \frac23 \\ \frac23 & -\frac13 \end{matrix}\right]\vt{20}{100}$ 
    \item

    \item
  \end{enumerate}
}

%\item
%\begin{enumerate}
%\item    
% Give $3$ vectors of length $1$ in $\R^3$ that are orthogonal to $\bu = \vd{1}{-1}{2}$.
%\item
% Which of the following sets are orthogonal? Orthonormal?
% If a set is only orthogonal, normalize its vectors to get an orthonormal set. 
% \[ A = \{ \vt{0.6}{0.8}, \vt{0.8}{-0.6} \}, \hspace{3cm} B = \{ \frac{1}{3}\vd{1}{-2}{2}, \frac{1}{\sqrt{18}} \vd{4}{1}{-1} \} \]
%\end{enumerate}


\end{enumerate}


\end{document}




