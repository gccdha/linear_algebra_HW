\documentclass[12pt,a4paper]{amsart}
\thispagestyle{empty}
\usepackage{lmodern}
\usepackage[
%scale=0.75,
margin=1in,
]{geometry}
\usepackage{versions}
\usepackage{mdframed}
\newif\ifsol\solfalse

%\soltrue % comment to hide solution

\ifsol \newenvironment{solution}{\smallskip \par\noindent\textbf{Solution:}}{\hfill\qed\smallskip} 
\else \excludeversion{solution} \fi
\ifsol \newenvironment{grading}{\par\noindent\textbf{Grading:}}{\hfill\\}
\else \excludeversion{grading} \fi

\newcommand{\ba}{\mathbf{a}}
\newcommand{\bb}{\mathbf{b}}
\newcommand{\be}{\mathbf{e}}
\newcommand{\bo}{\mathbf{0}}
%\newcommand{\bp}{\mathbf{p}}
%\newcommand{\bq}{\mathbf{q}}
\newcommand{\bu}{\mathbf u}
\newcommand{\bv}{\mathbf v}
\newcommand{\bw}{w}
\newcommand{\bx}{x}
\newcommand{\by}{y}
\newcommand{\Q}{\mathbb{Q}}
\newcommand{\R}{\mathbb{R}}
\newcommand{\Z}{\mathbb{Z}}
\newcommand\sol[1]{
\medskip
\begin{mdframed}
\textbf{Ans:\\} #1
\end{mdframed}
\medskip
}


\DeclareMathOperator{\Span}{Span}
\DeclareMathOperator{\Nul}{Nul}
\DeclareMathOperator{\Col}{Col}


\newcommand{\vt}[2]{\left[\begin{matrix} #1 \\ #2 \end{matrix}\right]}
\newcommand{\vd}[3]{\left[\begin{matrix} #1 \\ #2 \\ #3 \end{matrix}\right]}



\begin{document}
\begin{center}
{\Large\textbf{Math 2135 - Assignment 5}}\\
\medskip
Due October 4, 2024 \\
\end{center}
\smallskip
\thispagestyle{empty}


{\bf Problems 1-5 are review material for the first midterm on September 29. Solve them before Wednesday!}

%\smallskip

\begin{enumerate}
\item
 Let
\[ A = \left[\begin{matrix} 0 & 3 & 1 & 2 \\ 1 & 4 & 0 & 7 \\ 2 & -1 & -3 & 8 \end{matrix}\right], b = \vd{6}{5}{-8} \]
\begin{enumerate}
\item Give the solution for $Ax=b$ in parametrized vector form.
\item Give vectors that span the null space of $A$.
\end{enumerate}

\sol{
  \begin{enumerate}
    \item
    \begin{itemize}
      \item Row reduce augmented matrix:\\
        $$ \left[\begin{matrix}
        1 & 0 & -\frac{4}{3} & \frac{13}{3} & -3 \\
        0 & 1 & \frac{1}{3} & \frac{2}{3} & 2 \\ 
        0 & 0 & 0 & 0 & 0 
        \end{matrix}\right] $$
      \item Parametric form:\\
        $$x_1 -\frac{4}{3}x_3+\frac{13}{3}x_4 = -3, x_2 + \frac{1}{3}x_3 + \frac{2}{3}x_4 = 2$$ we have the two free variables $x_4=t, x_3=u$
        $$x = \left[\begin{matrix} -3 \\ 2 \\ 0 \\ 0 \end{matrix}\right] 
        +    t\left[\begin{matrix} -\frac{13}{3} \\ -\frac{2}{3} \\ 0 \\ 1 \end{matrix}\right] 
        +    u\left[\begin{matrix} \frac{4}{3} \\ -\frac{1}{3} \\ 1 \\ 0 \end{matrix}\right] $$
    \end{itemize}
    \item
      $\left[\begin{matrix} -\frac{13}{3} \\ -\frac{2}{3} \\ 0 \\ 1 \end{matrix}\right]$ and 
        $\left[\begin{matrix} \frac{4}{3} \\ -\frac{1}{3} \\ 1 \\ 0 \end{matrix}\right]$ span the nullspace of A
        because the solution to the equation $Ax=0$ is the same as the one in part a exept there is no constant term.
  \end{enumerate}
}
 

\item
 Let $T\colon \R^2\to\R^2$ be a linear transformation with
\[ T(\vt{1}{2}) = \vd{2}{-1}{1}\text{ and } T(\vt{3}{4}) = \vd{0}{1}{-2}. \]
 What is the standard matrix of $T$?

\sol{
  $T(\vt{3}{4}-2T(\vt{1}{2})=T(\vt{1}{0})=\vd{0}{1}{-2}-\vd{4}{-2}{2}=\vd{-4}{-1}{-4}$\\
  $\frac{1}{2}(T(\vt{1}{2}-T(\vt{1}{0}))=T(\vt{0}{1})=\frac{1}{2}(\vd{2}{-1}{1}-\vd{-4}{-1}{-4})=\vd{3}{0}{\frac{5}{2}}$\\
  so the standard matrix is $\left[\begin{matrix} -4 & 3 \\ -1 & 0 \\ -4 & \frac{5}{2} \end{matrix}\right]$

}
 
\item
 Let $T\colon \R^n\to\R^n, x\mapsto Ax$, be a surjective linear map. Show that $T$ is injective as well.  
 \sol{ \begin{itemize}
     \item Since the function is mapping between two sets of the same size ($\R^n$) the standar matrix A will be a squre $n\times n$ matrix.\\
     \item Because the function is surjective, every column must be linearly independant in order to make the range and codomain equal.\\
     \item This means that there has to be a pivot in every row.\\
     \item In a square matrix, this must mean that there is no column without a pivot, which means there are no free variables.\\
     \item Since there are no free variables, the function must be injective as well as surjective.
 \end{itemize} 
}

\item
 True or false? Explain your answer.
\begin{enumerate}
\item 
 If $Ax=b$ is inconsistent for some vector $b$, then $A$ cannot have a pivot in every column.
   
\item
  If vectors $\bv_1,\bv_2$ are linearly independent and $\bv_3$ is not in the span of $\bv_1,\bv_2$,
  then $\bv_1,\bv_2,\bv_3$ is linear independent.
\item
 The range of $T\colon\R^n\to\R^m, x\mapsto Ax,$ is the span of the columns of $A$.  
\end{enumerate}

\sol{
  \begin{enumerate}
    \item FALSE\\
    \item TRUE\\
    \item TRUE\\
  \end{enumerate}
}
  
\item  
\begin{enumerate}
\item
 Give examples of square matrices $A, B$ such that neither $A$ nor $B$ is $0$ (the matrix with all entries $0$) but
 $AB = 0$. 
\item
 If the first two columns of a matrix $B$ are equal, what can you say about the columns of $AB$?
\item
 We can view vectors in $\R^n$ as $n\times 1$ matrices. For
 $\bu = \left[\begin{matrix} 2  \\ -1  \\ 3 \end{matrix}\right], \bv = \left[\begin{matrix} 0  \\ 2 \\ 1 \end{matrix}\right]$
 compute $\bu^T\cdot \bv$ and $\bu\cdot \bv^T$. Interpret the results.  
\end{enumerate}

\sol{
  \begin{enumerate}
    \item
      $\left[\begin{matrix} 1 & -1 \\ 1 & -1\end{matrix}\right]\left[\begin{matrix} 1 & 1 \\ 1 & 1\end{matrix}\right]=\left[\begin{matrix} 0 & 0 \\ 0 & 0\end{matrix}\right]$
    \item
      We know that AB's first two columns will also be the same.
    \item
      $\left[\begin{matrix} 2 & -1 & 3\end{matrix}\right]\cdot\left[\begin{matrix} 0 \\ 2 \\ 1\end{matrix}\right]=\left[\begin{matrix} 1 \end{matrix}\right]$ \\ %TODO add explanation
      $\left[\begin{matrix} 2 \\ -1 \\ 3\end{matrix}\right]\cdot\left[\begin{matrix} 0 & 2 & 1\end{matrix}\right]=\left[\begin{matrix} 0 & 4 & 2 \\ 0 & -2 & -1 \\ 0 & 6 & 3 \end{matrix}\right]$ 
  \end{enumerate}
}
 
\item
 Prove for $A = \left[\begin{matrix} a & b \\ c & d \end{matrix}\right]$
 with $ad-bc\neq 0$ that
\vspace{-1mm} 
\[ A^{-1} = \frac{1}{ad-bc} \left[\begin{matrix} d & -b \\ -c & a \end{matrix}\right]. \] 
 Hint: Multiply $A$ with the given matrix and check the result.




\item
 Are the following invertible? Give the inverse if possible. 
\[ A = \left[\begin{matrix} 2 & 1 \\ 4 & -9 \end{matrix}\right],\quad
  B = \left[\begin{matrix} 2 & -3 \\ 4 & -6 \end{matrix}\right], \quad
  C = \left[\begin{matrix} 0 & 1 & 3 \\ 0 & 0 & 1 \\ 0 & -1 & -1 \end{matrix}\right]
\]



\item
 A {\bf diagonal matrix} $A$ has all entries $0$ except on the diagonal, that is,
 \[ A = \left[\begin{matrix} a_{11} & 0 &  \dots & 0 \\ 0 & a_{22} & \dots & 0 \\ \vdots & \vdots & \ddots & \vdots \\
     0 & 0 & \dots & a_{nn} \end{matrix}\right]. \]
 Under which conditions is $A$ invertible and what is $A^{-1}$? 

\end{enumerate}
\end{document}





