\documentclass[12pt,oneside]{amsart}
\thispagestyle{empty}
\usepackage{lmodern}
\usepackage[
%scale=0.75,
margin=0.95in,
]{geometry}
\usepackage{versions}
\newif\ifsol\solfalse

%\soltrue % comment to hide solution

\ifsol \newenvironment{solution}{\par\noindent\textbf{Solution:\\}}{\hfill\qed\\} 
\else \excludeversion{solution} \fi

%\newcommand{\ba}{\mathbf{a}}
%\newcommand{\bb}{\mathbf{b}}
%\newcommand{\be}{\mathbf{e}}
\newcommand{\bo}{\mathbf{0}}
%\newcommand{\bp}{\mathbf{p}}
%\newcommand{\bq}{\mathbf{q}}
\newcommand{\bu}{\textbf{u}}
\newcommand{\bv}{\textbf{v}}
\newcommand{\bw}{\textbf{w}}
\newcommand{\bx}{x}
\newcommand{\by}{y}
\newcommand{\Q}{\mathbb{Q}}
\newcommand{\R}{\mathbb{R}}
\newcommand{\Z}{\mathbb{Z}}

\DeclareMathOperator{\Span}{Span}
\DeclareMathOperator{\Nul}{Nul}
\DeclareMathOperator{\Col}{Col}
\DeclareMathOperator{\Row}{Row}
\DeclareMathOperator{\proj}{proj}

\usepackage{mdframed}
\newcommand
\sol[1]{
\medskip
\begin{mdframed}
\textbf{Ans:\\} #1
\end{mdframed}
\medskip
}

\newcommand{\vt}[2]{\left[\begin{matrix} #1 \\ #2 \end{matrix}\right]}
\newcommand{\vd}[3]{\left[\begin{smallmatrix} #1 \\ #2 \\ #3 \end{smallmatrix}\right]}
\newcommand{\vv}[4]{\left[\begin{smallmatrix} #1 \\ #2 \\ #3 \\ #4 \end{smallmatrix}\right]}



\begin{document}
\begin{center}
{\Large\textbf{Math 2135 - Assignment 13}}\\
\medskip
Due December 9, 2024 \\ Maxwell Rodgers
\end{center}
\bigskip
\thispagestyle{empty}


\begin{enumerate} 
\item
\begin{enumerate}
\item    
 Give $3$ vectors of length $1$ in $\R^3$ that are orthogonal to $\bu = \vd{1}{-1}{2}$.

\item
 Which of the following sets are orthogonal? Orthonormal?
% If a set is only orthogonal, normalize its vectors to get an orthonormal set. 
 \[ A = \{ \vt{0.6}{0.8}, \vt{0.8}{-0.6} \}, \hspace{3cm} B = \{ \frac{1}{3}\vd{1}{-2}{2}, \frac{1}{\sqrt{18}} \vd{4}{1}{-1} \} \]
\end{enumerate}

\sol{\begin{enumerate}
    \item We need to solve the equation $x-y+2z=0$ by setting each variable equal to zero in turn, we can easily find orthogonal vectors:
  $\vd{0}{2}{1},\vd{2}{0}{-1},\vd{-2}{0}{1}$ and now we just have to normalize them: $\frac{\sqrt5}{5}\vd{0}{2}{1},\frac{\sqrt5}{5}\vd{2}{0}{-1},\frac{\sqrt5}{5}\vd{-2}{0}{1}$ 
  \item
  Set A is orthonormal because $\vt{0.6}{0.8} \cdot \vt{0.8}{-0.6} = 0.48-0.48 = 0$(orthogonal) and both vectors have length $\sqrt{0.36+0.64}=1$(normal).\\
  Set B is also orthonormal because $\vd{1}{-2}{2}\cdot\vd{4}{1}{-1}=4-2-2=0$ (orthogonal). and both have length of 1. $\frac{\sqrt{1+4+4}}{3}=\frac33=1,\frac{\sqrt{16+1+1}}{\sqrt{18}}=\frac{\sqrt{18}}{\sqrt{18}}=1$
\end{enumerate}
}


\item
\begin{enumerate}
\item    
 Let $W$ be the subspace of $\R^3$ with orthonormal basis
 $B = ( \frac{1}{3}\vd{2}{-1}{2}, \frac{1}{\sqrt{5}} \vd{1}{2}{0} )$.
 Compute the coordinates $[x]_B$ for $x =\vd{7}{4}{4}$ in $W$ using dot products.
\item
 Give a basis for $W^\perp$.
\item
 Find the closest point to $y = \vd{1}{2}{3}$ in $W$. What is the distance from $y$ to $W$?
\end{enumerate}

\sol{
  \begin{enumerate}
    \item
      $[x]_B=\vt{x\cdot\bu_1}{x\cdot\bu_2}=\vt{\frac13((7)(2)+(4)(-1)+(2)(4))}{\frac{1}{\sqrt5}((7)(1)+(4)(2)+(4)(0))}
      =\vt{6}{3\sqrt5}$
    \item
      Since $W$ is 2 dimensional and in $\R^3$ we know that $W^\perp$ is 1 dimensional. we can use the cross product of the basis vectors of $W$:
      $\vd{2}{-1}{2}\times\vd120=\vd{-4}{2}{5}=$basis of $W^\perp$.
    \item
      We find the point at the tip of the projection of y onto W:
      $\frac19\vd2{-1}2(\vd123\cdot\vd2{-1}2)+\frac15\vd120(\vd123\cdot\vd120)=\frac23\vd2{-1}2+\vd120=
      \frac13\vd{7}{4}{4} = (\frac73,\frac43,\frac43)$\\
      The distance from y to this point is $|\frac13(\vd369-\vd744)|=\frac13\sqrt{-4^2+2^2+5^2}=\sqrt5$
  \end{enumerate}
}

\item
 True or false. Explain your answers.
\begin{enumerate}
\item Every orthogonal set is also orthonormal.
\item Not every orthonormal set in $\R^n$ is linearly independent.
\item For each $\bx$ and each subspace $W$, the vector $\bx-\proj_W(\bx)$ is orthogonal to $W$.
\end{enumerate}

\sol{
  \begin{enumerate}
    \item FALSE\\
      It is possible for vectors to be perpendicular to each other and not have magnitude of 1.
    \item FALSE\\
      In order to be an orthonormal set all of the vectors must be perpendicular to each other which means that they cannot be linear combinations of any of the other vectors.
    \item TRUE\\
      This is just the same as representing x as the sum of 2 vectors, one of them being in W, forcing the other one to be perpendicular to the first and therefore perpendicular to the whole of W.
  \end{enumerate}
}


\item
 Let $W$ be a subset of $\R^n$. Show that its orthogonal complement
\[ W^\perp := \{ x\in\R^n \ |\ x \text{ is orthogonal to all } w\in W \} \]
 is a subspace of $\R^n$.

 \sol{
  Take $\bu\in W\hspace{2mm}\bv,\bw\in W^\perp\hspace{2mm}c\in\R$\\
  \begin{itemize}
 \item Since $\bv$ and $\bu$ are in perpendicular spaces, they must be perpendicular to each other, and thus
  $\bu\cdot\bv=0$. From the properties of dot products we can see that $\bu\cdot c\bv=c(\bu\cdot\bv)=0(c)=0$
  The same can be said for $\bu$ and $\bw$. This means that $c\bv\in W^\perp$
  \item We can also see then that $\bu\cdot(\bw+\bv)=\bu\cdot\bw+\bu\cdot\bv=0+0=0$ so $(\bw+\bv)\in W^\perp$
\end{itemize}
Thus $W^\perp$ is a subspace of $\R^n$ because all of its elements are from $\R^n$, and it is closed under addition and scalar multiplication.
}

\item
 Let $W$ be a subspace of $\R^n$. Show that 
\begin{enumerate}
\item $W \cap W^\perp = 0$
\item $\dim W + \dim W^\perp = n$  \\
 Hint: Let $w_1,\dots, w_k$ be a basis of $W$. Use that $x\in W^\perp$ iff $x$ is orthogonal to $w_1,\dots, w_k$.
\end{enumerate}

\sol{
  \begin{enumerate}
    \item
      Since all of the elements in $W^\perp$ have to be perpendicular to the elements in $W$, the only element that they could share would be an element that is perpedicular to itself.
      This would mean that the dot product between this element and itself would have to be zero, meaning that it must be the zero vector $0$.
    \item
      Take an orthogonal basis of $\R^n$, $(u_1,\dots,u_n)$. This basis can be split into two parts by chosing
      any subset and its compliment. Each of these subsets is an orthogonal basis of a subspace.
      Additionaly, the subspaces defined by these basis would be perpendicular because every basis
      vector in each one would be orthogonal to every basis vector in the other (because we started with an orthogonal basis).
      Thus we get two subspaces $W$ and $W^\perp$. Since the bases of theses subspaces have a total
      of $n$ elements between them, $\dim W +\dim W^\perp=n$.
      
  \end{enumerate}
}



\item
  Find the least squares solutions of $Ax=b$.

\noindent (a) $A=\begin{bmatrix} 1 & 3 \\ 1 & -1 \\ 1 & 1 \end{bmatrix}$, $b = \vd{5}{1}{0}$
 \hspace{3cm} (b)  $A=\begin{bmatrix} 1 & 1 & 0 \\ 1 & 1 & 0  \\ 1 & 0 & 1 \\ 1 & 0 & 1 \end{bmatrix}$, $b = \vv{-1}{2}{-3}{4}$  

 \sol{ 
   From theorem 13 we know that the least squares solutions can be found by solving $A^TAx=A^Tb$
\begin{enumerate}
  \item
    $A^TA=\left[\begin{matrix}1&1&1\\3&-1&1\end{matrix}\right]\left[\begin{matrix}1&3\\1&-1\\1&1\end{matrix}\right]
    = \left[\begin{matrix}3&3\\3&11\end{matrix}\right]$\\
    $A^Tb=\left[\begin{matrix}1&1&1\\3&-1&1\end{matrix}\right]\vd510=\vt{6}{14}$\\
    So now we can row reduce the augmented matrix $\left[\begin{matrix}3&3&6\\3&11&14\end{matrix}\right]
    \to \left[\begin{matrix}3&3&6\\0&8&8\end{matrix}\right]\to\left[\begin{matrix}1&0&1\\0&1&1\end{matrix}\right]$ so $x=\vt11$

  \item
  $A^TA=\left[\begin{matrix}1&1&1&1\\1&1&0&0\\0&0&1&1\end{matrix}\right]\left[\begin{matrix}1&1&0\\1&1&0\\1&0&1\\1&0&1\end{matrix}\right]=
  \left[\begin{matrix}4&2&2\\2&2&0\\2&0&2\end{matrix}\right]$\\
  $A^Tb=\left[\begin{matrix}1&1&1&1\\1&1&0&0\\0&0&1&1\end{matrix}\right]\vv{-1}{2}{-3}{4}=\vd211$
  \\So now we row reduce the augmented matrix $\left[\begin{matrix}4&2&2&2\\2&2&0&1\\2&0&2&1\end{matrix}\right]\to\left[\begin{matrix}2&1&1&1\\0&1&-1&0\\0&0&0&0\end{matrix}\right]$
  So $x=\vd{\frac12}{0}{0}+x_3\vd{-1}{1}{1}$
\end{enumerate}
 }
\item
 True or false for $A\in\R^{m\times n}$ and $b\in\R^m$. Explain your answers.
\begin{enumerate}
\item A least squares solution of $Ax=b$ is an $\hat{x}$ such that $A\hat{x}$ is as close as possible to $b$.
\item A least squares solution of $Ax=b$ is an $\hat{x}$ such that $A\hat{x} = \hat{b}$ for $\hat{b}$
  the orthogonal projection of $b$ onto $\Col A$.
\item The point in $\Col A$ closest to $b$ is a least squares solution of $Ax=b$.  
\item If $Ax=b$ is consistent, then every solution $x$ is a least squares solution.
\end{enumerate}

\sol{
  \begin{enumerate}
    \item TRUE\\
      Least squares means that the euclidian distance between the solution's output and the desired output is minimized.
    \item TRUE\\
      Since a) is true this is also true because $\hat{b}$ is the closest possible solution to $b$.
    \item TRUE\\
      See two explanations above.
    \item TRUE\\
      The difference can't be less than 0 so I guess this is true.
  \end{enumerate}
}
\end{enumerate}


\end{document}




