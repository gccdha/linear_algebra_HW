\documentclass[12pt,oneside]{amsart}
\thispagestyle{empty}
\usepackage{lmodern}
\usepackage[
%scale=0.75,
margin=1in,
]{geometry}
\usepackage{versions}
\newif\ifsol\solfalse

%\soltrue % comment to hide solution

\ifsol \newenvironment{solution}{\par\noindent\textbf{Solution:\\}}{\hfill\qed\\} 
\else \excludeversion{solution} \fi
\ifsol \newenvironment{grading}{\par\noindent\textbf{Grading:}}{\hfill\\}
\else \excludeversion{grading} \fi

\newcommand{\id}{P}
\newcommand{\bb}{b}
\newcommand{\be}{e}
\newcommand{\bo}{\mathbf{0}}
%\newcommand{\bp}{\mathbf{p}}
%\newcommand{\bq}{\mathbf{q}}
\newcommand{\bu}{u}
\newcommand{\bv}{v}
\newcommand{\bw}{w}
\newcommand{\bx}{x}
\newcommand{\by}{y}
\newcommand{\Q}{\mathbb{Q}}
\newcommand{\R}{\mathbb{R}}
\newcommand{\Z}{\mathbb{Z}}

\usepackage{mdframed}
\newcommand
\sol[1]{
\medskip
\begin{mdframed}
\textbf{Ans:\\} #1
\end{mdframed}
\medskip
}


\DeclareMathOperator{\Span}{Span}
\DeclareMathOperator{\Nul}{Nul}
\DeclareMathOperator{\Col}{Col}
\DeclareMathOperator{\rank}{rank}
\DeclareMathOperator{\Row}{Row}


\newcommand{\vt}[2]{\left[\begin{matrix} #1 \\ #2 \end{matrix}\right]}
\newcommand{\vd}[3]{\left[\begin{smallmatrix} #1 \\ #2 \\ #3 \end{smallmatrix}\right]}



\begin{document}
\begin{center}
{\Large\textbf{Math 2135 - Assignment 9}}\\
\medskip
Completed October 31, 2024 \\ Maxwell Rodgers
\end{center}
\bigskip
\thispagestyle{empty}

\begin{enumerate}
\item
Let $\bb_1 = \vd{1}{2}{-1}$, $\bb_2=\vd{1}{1}{3}$, $\bb_3=\vd{1}{2.5}{-5}$.
\begin{enumerate}
\item Find vectors $\bu_1,\ldots,\bu_k$ such that $(\bb_1,\bb_2,\bu_1,\ldots,\bu_k)$ is a basis for $\R^3$.
\item Find vectors $\bv_1,\ldots,\bv_\ell$ such that $(\bb_3,\bv_1,\ldots,\bv_\ell)$ is a basis for $\R^3$.
\end{enumerate}
Prove that your choices for (a) and (b) form a basis.


\sol{
  \begin{enumerate}
    \item
      The only vector that is needed to make a basis with $b_1$ and $b_2$ is $b_1\times b_2=\vd{7}{-4}{1}=u_1$
      We know this is a basis because there are three vectors and they are linearly independant. We can see this
      in the row reduction of the matrix $\left[\begin{matrix}1 & 1 & 7 \\ 2 & 1 & -4 \\ -1 & 3 & 1\end{matrix}\right] \to \left[\begin{matrix}1 & 1 & 7 \\ 0 & -1 & -18\\ 0 & 0 & -64\end{matrix}\right]$ because there is a pivot in every column.
    \item
      We need two linearly independant vectors who's spanned plane does not conatin $b_3$. The vectors $\vd{1}{0}{0}$ and $\vd{0}{1}{0}$ fit this description.
      This is a basis because there are three vectors and they are linearly independant. We can see this
      by looking at the matrix of the basis $\left[\begin{matrix} 1 & 0 & 1 \\ 0 & 1 & 2.5 \\ 0 & 0 & -5\end{matrix}\right]$ which is already in echelon form and has a basis in every column.
  \end{enumerate}}

\item
A $25\times 35$ matrix $A$ has $20$ pivots.
Find $\dim\Nul A$, $\dim\Col A$, $\dim\Row A$, and $\rank A$.

\sol{$\dim\Nul A = 35-20 = 15 \\
      \dim\Col A = \dim\Row A = \rank A = 20$}

\item True or false? Explain.
\begin{enumerate}
\item A basis of $B$ is a set of linear independent vectors in $V$ that is as large as possible.
\item If $\dim V = n$, then any $n$ vectors that span $V$ are linearly independent.
\item Every 2-dimensional subspace of $\R^2$ is a plane. 
\end{enumerate}

\sol{\begin{enumerate}
  \item True\\
    A basis must be linearly independant and have $n$ vectors for an $n$ dimensional space. A set of linearly independant vectors in $n$ dimensional space V can have at most $n$ vectors.
  \item True\\
    If $n$ vectors are not linearly independant then the space they span is less that $n$ dimensional so they cannot span $V$. %contrapositive
  \item True\\
    Every 2 dimensional subspace of $\R^2$ is just $\R^2$, but $\R^2$ is a plane so technicaly this is true.
\end{enumerate}}

  
\item
 Let $P_3$ the vector space of polynomials of degree $\leq 3$ over $\R$ with basis $B = (1,x,x^2,x^3)$.  
\begin{enumerate}
\item
 Find the matrix $d_{B\leftarrow B}$ for the derivation map  $d\colon P_3\to P_3, p\to p'$.   
\item
 Use $d_{B\leftarrow B}$ to compute $[p']_B$ and $p'$ for the polynomial $p$ with $[p]_B = (-3,2,0,1)$. 
\end{enumerate} 

\sol{
  \begin{enumerate}
    \item
      Since we know how each of the unit vectors transform (what their derivatives are) we can easily make a matrix: $\left[\begin{matrix} 0 & 1 & 0 & 0 \\ 0 & 0 & 2 & 0 \\ 0 & 0 & 0 & 3 \\ 0 & 0 & 0 & 0 \end{matrix}\right]$
    \item
      $d_{B\leftarrow B}[p]_B = [p']_B = (2, 0, 3, 0)$ which means that $p'=2+3x^2$
  \end{enumerate}
}

\item 
Let $B=(\vt11,\vt1{-1})$ and $C=(\vt25,\vt13)$ be bases of $\R^2$, let $E$ be the standard basis of $\R^2$.
\begin{enumerate}
\item Find the change of coordinates matrix $P_{E\leftarrow B}$ for $f:\ [ \bu ]_B \mapsto [\bu]_E$.
\item Find the change of coordinates matrix $P_{C\leftarrow E}$ for $g:\ [ \bu ]_E \mapsto [\bu]_C$.
\item Find the change of coordinates matrix $P_{C\leftarrow B}$ for $h:\ [ \bu ]_B \mapsto [\bu]_C$.
  
Hint: $h$ is the composition of $g$ and $f$, $h([ \bu ]_B) = g(f([ \bu ]_B))$.
\end{enumerate}

\sol{ 
  \begin{enumerate}
    \item
      This is just the basis vectors of B in a matrix because $[b_1]_B=e_1$.
      $\left[\begin{matrix}1 & 1 \\ 1 & -1 \end{matrix}\right]$
    \item
      This is just the inverse of the matrix of the basis vectorss of C: $\left[\begin{matrix} 2 & 1 \\ 5 & 3\end{matrix}\right]^{-1}=\left[\begin{matrix} 3 & -1 \\ -5 & 2\end{matrix}\right]$
    \item
    Since this is the composition of g and f we can just multiply the previously found matrices:\\
      $\left[\begin{matrix} 3 & -1 \\ -5 & 2 \end{matrix}\right]\cdot\left[\begin{matrix} {1} & {1} \\ {1} & -{1} \end{matrix}\right]= \left[\begin{matrix} 2 & 4\\ -3 & -7  \end{matrix}\right]$
  \end{enumerate}
}

\item
 Determine the standard matrix for the reflection $t$ of $\R^2$ at the line $3x+y=0$ as follows:
\begin{enumerate}
\item Find a basis $B$ of $\R^2$ whose vectors are easy to reflect.
\item Give the matrix $t_{B\leftarrow B}$ for the reflection with respect to the coordinate system determined by $B$.
\item Use the change of coordinate matrix to compute the standard matrix $t_{E\leftarrow E}$ with respect to the 
 standard basis $E = (e_1,e_2)$.
\end{enumerate}

\sol{   \begin{enumerate}
    \item
      A vector on the line ($b_1=\vt1{-3}$) and a vector orthogonal to the line ($b_2=\vt31$) form a basis that is easy to reflect.
    \item
      We find that since $T(b_1)=b_1$ and $T(b_2)=-b_2$, these end up as $e_1$ and $-e_2$ when transformed to B. This leaves us with the matrix $t=\left[\begin{matrix} 1 & 0 \\ 0 & -1 \end{matrix}\right]$
    \item
      To do this, we need a matrix to go from E to B, then B to B, then B to E (right to left). the B to E and E to B matrices will be inverses of each other and the B to E matrix will just be the basis vectors.
      $t_{E\leftarrow E} = P_{E\leftarrow B}T_{B\leftarrow B}P_{B\leftarrow E} =\left[\begin{matrix}1 & 3 \\ -3 & 1\end{matrix}\right] \left[\begin{matrix}1 & 0 \\ 0 & -1\end{matrix}\right]\left[\begin{matrix} \frac1{10} & -\frac3{10} \\ \frac3{10} & \frac1{10} \end{matrix}\right]  = \left[\begin{matrix} -\frac45 & -\frac35 \\ -\frac35 & \frac45 \end{matrix}\right]$
  \end{enumerate} 
}

\item
\begin{enumerate}
\item
 Determine the standard matrix $A$ for the rotation $r$ of $\R^3$ around the $z$-axis through the angle $\pi/3$
 counterclockwise.

 Hint: Use the matrix for the rotation around the origin in $\R^2$ for the $xy$-plane.
 What happens to $e_3$ under this rotation?

\item
 Consider the rotation $s$ of $\R^3$ around the line spanned by $\vd{1}{2}{3}$ through the angle $\pi/3$
 counterclockwise. Find a basis of $\R^3$ for which the matrix $s_{B\leftarrow B}$ is equal to $A$ from (a).
\item
 Give the standard matrix $s_{E\leftarrow E}$ for the standard basis $E$ (You do not need to actually multiply and invert the involved matrices; the product formula is enough). 
\end{enumerate}

\sol{
  \begin{enumerate}
    \item
      The matrix for the rotation arround the origin of $\frac{\pi}{3}$ radians is $\left[\begin{matrix} \frac12 & \frac{\sqrt3}{2} \\  -\frac{\sqrt3}{2} & \frac12 \end{matrix}\right]$ from this we can extrapolate to the rotation arround the z axis \\
      x and y don't depend on z and z just remains the same (because $e_3$ does not change): $\left[\begin{matrix} \frac12 & \frac{\sqrt3}{2} & 0 \\  -\frac{\sqrt3}{2} & \frac12 & 0 \\ 0 & 0 & 1 \end{matrix}\right]$
    \item
      We know that $[e_3]_B=\vd123$ so this means we just need to transform the other unit vectors in the same way. Since all the vectors started the same length they also have to be the same length after the transform. (???)
    \item
      ...
\end{enumerate}
}

\item
 The \emph{kernel} of a linear map $h\colon V\to W$ is the subspace of $V$, 
 \[ \{ v\in V \ |\ h(v) = 0 \}. \]
\begin{enumerate}
\item
 Determine the kernel and the image of $d\colon P_3\to P_3, p\to p'$.
\item
 Is $d$ injective, surjective, bijective?
\end{enumerate}  

\sol{
  \begin{enumerate}
    \item
      The kernel in this case is just the polynomials that only have a constant term (because the derivative of a constant is zero), so the basis is just the vector $\left[\begin{matrix}1\\0\\0\\0\end{matrix}\right]$\\
      The image is just $P_2$ because these are all the polynomials that you can get by deriving polynomials in $P_3$
    \item
      No, the range ($P_2$) is smaller than the codomain and the domain (both $P_3$) so it cannot be injective or surjective.
  \end{enumerate}
}

\end{enumerate}


\end{document}



