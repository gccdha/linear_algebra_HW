\documentclass[12pt,a4paper]{amsart}
\thispagestyle{empty}
\usepackage{lmodern}
\usepackage[
%scale=0.75,
margin=1in,
]{geometry}
\usepackage{versions}
\usepackage{mdframed}
\usepackage{ wasysym }
\newif\ifsol\solfalse

%\soltrue % comment to hide solution

\ifsol \newenvironment{solution}{\smallskip \par\noindent\textbf{Solution:}}{\hfill\qed\smallskip} 
\else \excludeversion{solution} \fi
\ifsol \newenvironment{grading}{\par\noindent\textbf{Grading:}}{\hfill\\}
\else \excludeversion{grading} \fi

\newcommand{\ba}{\mathbf{a}}
\newcommand{\bb}{\mathbf{b}}
\newcommand{\be}{\mathbf{e}}
\newcommand{\bo}{\mathbf{0}}
%\newcommand{\bp}{\mathbf{p}}
%\newcommand{\bq}{\mathbf{q}}
\newcommand{\bu}{u}
\newcommand{\bv}{\mathbf v}
\newcommand{\bw}{w}
\newcommand{\bx}{x}
\newcommand{\by}{y}
\newcommand{\Q}{\mathbb{Q}}
\newcommand{\R}{\mathbb{R}}
\newcommand{\Z}{\mathbb{Z}}
\newcommand\sol[1]{
\medskip
\begin{mdframed}
\textbf{Ans:\\} #1
\end{mdframed}
\medskip
}

\DeclareMathOperator{\Span}{Span}
\DeclareMathOperator{\Nul}{Nul}
\DeclareMathOperator{\Col}{Col}


\newcommand{\vt}[2]{\left[\begin{matrix} #1 \\ #2 \end{matrix}\right]}
\newcommand{\vd}[3]{\left[\begin{matrix} #1 \\ #2 \\ #3 \end{matrix}\right]}



\begin{document}
\begin{center}
{\Large\textbf{Math 2135 - Assignment 4}}\\
\medskip
Due September 27, 2024 \\
Completed by Maxwell Rodgers, September 24, 2024
\end{center}
\medskip
\thispagestyle{empty}

\begin{enumerate}
   

\item
 Let $T\colon\R^2\to\R^3$ be a linear map such that 
\[ T(\left[\begin{matrix} 1 \\ 2 \end{matrix}\right]) = \left[\begin{matrix} 2 \\ 0 \\ -3 \end{matrix}\right],\ T(\left[\begin{matrix} 3 \\ 2 \end{matrix}\right]) = \left[\begin{matrix} -2 \\ 2 \\ 1 \end{matrix}\right]. \]
\begin{enumerate}
\item
  Use linearity to find $T(e_1)$ and $T(e_2)$ for the unit vectors $e_1,e_2$ in $\R^2$.
\item
 Give the standard matrix for $T$ and determine $T(\left[\begin{matrix} x \\ y \end{matrix}\right])$ for arbitrary $x,y\in\R$.
\end{enumerate}


\sol{
  \begin{enumerate}
    \item
    \begin{itemize}
      \item $\frac{1}{2}(\left[\begin{matrix} 3 \\ 2 \end{matrix}\right]-\left[\begin{matrix} 1 \\ 2 \end{matrix}\right])=\left[\begin{matrix} 1 \\ 0 \end{matrix}\right]$\\
            $\frac{1}{2}(T(\left[\begin{matrix} 3 \\ 2 \end{matrix}\right])-T(\left[\begin{matrix} 1 \\ 2 \end{matrix}\right]))=T(\left[\begin{matrix} 1 \\ 0 \end{matrix}\right])$\\
            $\frac{1}{2}(\left[\begin{matrix} -2 \\ 2 \\ 1 \end{matrix}\right]+\left[\begin{matrix} -2 \\ 0 \\ 3 \end{matrix}\right])=T(\left[\begin{matrix} 1 \\ 0 \end{matrix}\right])=T(e_1)=\left[\begin{matrix} -2 \\ 1 \\ 2 \end{matrix}\right]$\\
      \item $-\frac{1}{4}(\left[\begin{matrix} 3 \\ 2 \end{matrix}\right]-3\left[\begin{matrix} 1 \\ 2 \end{matrix}\right])=\left[\begin{matrix} 0 \\ 1 \end{matrix}\right]$\\
            $-\frac{1}{4}(T(\left[\begin{matrix} 3 \\ 2 \end{matrix}\right])-3T(\left[\begin{matrix} 1 \\ 2 \end{matrix}\right]))=T(\left[\begin{matrix} 0 \\ 1 \end{matrix}\right])$\\
            $-\frac{1}{4}(\left[\begin{matrix} -2 \\ 2 \\ 1 \end{matrix}\right]+\left[\begin{matrix} -6 \\ 0 \\ 9 \end{matrix}\right])=T(\left[\begin{matrix} 0 \\ 1 \end{matrix}\right])=T(e_2)=\left[\begin{matrix} 2 \\ -\frac{1}{2} \\ -\frac{5}{2} \end{matrix}\right]$\\
    \end{itemize}
      \item
        Standard Matrix: $\left[\begin{matrix}-2&2\\1&-\frac{1}{2}\\2&-\frac{5}{2}\end{matrix}\right]$\\
        $T(\left[\begin{matrix} x \\ y \end{matrix}\right])=\left[\begin{matrix} 2y-2x \\ -\frac{y}{2}+x \\ -\frac{5y}{2}+2x \end{matrix}\right]$
  \end{enumerate}
}


\item
 Is the following injective, surjective, bijective? What is its range?
\[ T:\R^3\to\R^2,\ x \mapsto \left[\begin{matrix} 0 & 2 & -1 \\ 0 & 0 & 3 \end{matrix}\right]\cdot x \]

\sol{
  The matrix is already in echelon form. If we put it in reduced echelon form we get this: $\left[\begin{matrix} 0 & 1 & 0 \\ 0 & 0 & 1 \end{matrix}\right]$\\
  The columns of the matrix are the zero vector and unit vectors of $\R^2$, which of course span $\R^2$. This means that the function is surjective. Because one of the columns is the zero vector, the vectors cannot be linearly independant. Therefore the function $T$ is only Surjective. Since the function is surjective, the range is the codomain which is $\R^2$.
};

\item
 Is the following injective, surjective, bijective? %What is its range?
\[ T:\R^3\to\R^3,\ x \mapsto \left[\begin{matrix} 1 & -1 & 2 \\ -2 & 0 & 1 \\ 3 & -1 & 1 \end{matrix}\right]\cdot x \]

\sol{
  \begin{itemize}
    \item
      Row reduce:\\
      $\left[\begin{matrix} 1 & -1 & 2 \\ 0 & -2 & 5 \\ 0 & 0 & 0 \end{matrix}\right]$
    \item
      Since an entire row is zeroes, the columns of this matrix cannot span $\R^3$ and thus it isn't surjective. Since there are 3 columns and each only has 2 or fewer of the same rows with non-zero elements, the columns cannot linearly independant and thus $T$ cannot be injective.
    \item
      \textbf{The fuction is not injective or surjective.}
  \end{itemize}
}

\item %\cite[cf. Section 1.9, Ex 23/24]{La-LA}
  True or False? Explain why and correct the false statements to make them true.
\begin{enumerate}
\item
 A linear transformation $T\colon\R^n\to\R^m$ is completely determined by the images of the unit vectors
 in $\R^n$.
\item
 Not every linear transformation $T\colon\R^n\to\R^m$ can be written as $T(x)=Ax$ for some matrix $A$.
\item
 The composition of any two linear transformations is linear as well.
\end{enumerate}


\sol{
  \begin{enumerate}
    \item
      \textbf{True}\\
      Because the properties of linear transformations allow us to determine the output for any
      input that is a linear combination of inputs with known output, and the unit vectors of
      $\R^n$ span $\R^n$, the unit vectors are enough to determine the output of any input.
    \item
      \textbf{False}\\
      Every linear transformation can be writen as $T(x) = Ax$ for some matrix $A$\\
      If we have some vector $\bx$ we can write it as each element multiplied by the unit vector 
      for the coresponding row. If we then apply a linear transformation to this sum of scaled 
      vectors, and use the properties of linear transformations to seperate the vectors and pull 
      out the scalars, we get the elements of $\bx$ multiplied by the transformations of the unit
      vectors. This is equivilant to $A\bx$ if $A$'s columns are just the transformed unit vectors.
    \item
      \textbf{True}\\
      If we take any two linear transformations, given their domain and ranges are compatible we 
      can compose them. Since every linear transformation has an associated matrix $A$ we can show
      that the composition of the two linear maps has such a matrix. Take maps $T$ and $S$, arbitrary vector $\bx$, and composition  $T\ocircle S(\bx)$.
      The composition is equivilant to $A_T(A_S\bx)$. Because we already stipulated that the functions
      have compatible dimensions, matrix multiplication in this case is associative and the new
      matrix for the composed linear transformation is $A_{T\ocircle S(\bx)}=A_TA_S$.
  \end{enumerate}
}



\item %\cite[cf. Section 1.9, Ex 23/24]{La-LA}
  True or False? Explain why and correct the false statements to make them true.
\begin{enumerate}
\item
 $T:\R^n\to\R^m$ is onto $\R^m$ if every vector $\bx\in\R^n$ is mapped onto some vector in $\R^m$.
\item
 $T:\R^n\to\R^m$ is one-to-one if every vector $\bx\in\R^n$ is mapped onto a unique vector in $\R^m$.
\item
 A linear map $T:\R^3\to\R^2$ cannot be one-to-one.
\item 
 There is a surjective linear transformation $T\colon\R^3\to\R^4$. 
\end{enumerate}

\sol{
  \begin{enumerate}
    \item
      \textbf{False}\\
      This is just a description of a function, and says nothing of the range.\\
      $T:\R^n\to\R^m$ is onto $\R^m$ if every vector $\bx\in\R^n$ is mapped onto some vector in $\R^m$
      \textbf{and} every vector in $\R^m$ is mapped to by a vector in $\R^n$.
    \item
      \textbf{True}\\
      This is what one-to-one means. Every possible input has a unique output.
    \item
      \textbf{True}\\
      The associated matrix to the map $T$ would have to be a $2\times3$ matrix. This ensures that there would be at least one free variable meaning that the function could not be one-to-one.
    \item
      \textbf{False}\\
      $T$ is surjective iff its columns span $\R^4$. There are only 3 columns since the matrix must map from $\R^3$ and 3 vectors cannot span $\R^4$.
  \end{enumerate}
}


\item
 If defined, compute the following for the matrices
 \[
    A = \left[ \begin{array}{ccc} 2 & 1 & -4 \\ 3 & -1 & 1 \end{array} \right],\
    B = \left[ \begin{array}{cc} 1 & -1 \\ -3 & 4 \\ 2 & 0 \end{array} \right],\
    C = \left[ \begin{array}{cc} -2 & 1 \\ 2 & -3 \end{array} \right] \]
 Else explain why the computation is not defined. 

\noindent (a) $AB$ \hfill (b) $BA$ \hfill (c) $AC$ \hfill (d) $A+C$  \hfill (e) $AB+2C$  


\sol{
  \begin{enumerate}
    \item
      $\left[ \begin{array}{cc} 2-3-8 & -2+4+0 \\ 3+3+2 & -3-4+0 \end{array} \right]=\left[ \begin{array}{cc} -9 & 2 \\ 8 & -7 \end{array} \right]$
    \item
    $\left[ \begin{array}{ccc} 2-3 & 1+1 & -4-1 \\ -6+12 & -3-4 & 12+4 \\ 4+0 & 2+0 & -8+0 \end{array} \right] = \left[ \begin{array}{ccc} -1 & 2 & -5 \\ 6 & -7 & 16 \\ 4 & 2 & -8 \end{array} \right]$
    \item
      Not possible. The height of C is not the same as the width of A.
    \item
      Not possible. A and C are not the same size.
    \item
      $\left[ \begin{array}{cc} -9-4 & 2+2 \\ 8+4 & -7-6 \end{array} \right]=\left[ \begin{array}{cc} -13 & 4 \\ 12 & -13 \end{array} \right]$
  \end{enumerate}
}

\item
 Let $T\colon\R^2\to\R^2$ first rotate points around the origin by $60^\circ$ counter clockwise and then reflect
 points at the line with equation $y=x$. Give the standard matrix for $T$.

\begin{enumerate}
\item Recall the standard matrix $A$ for the rotation $R$ by  $60^\circ$ from class.  
\item  Determine the standard matrix $B$ for the reflection $S$ at the line with equation $y=x$ (a sketch will help).
\item Since $T$ is the composition of $S$ and $R$, compute the standard matrix $C$ of $T$ as the product of $B$ and $A$.
  Careful about the order!
\end{enumerate}

\sol{
  \begin{enumerate}
    \item
      $A=\left[\begin{matrix} \frac{1}{2} & -\frac{\sqrt3}{2} \\ \frac{\sqrt3}{2} & \frac{1}{2} \end{matrix}\right]$
    \item
      $B=\left[\begin{matrix} 0 & 1 \\ 1 & 0 \end{matrix}\right]$
    \item
      $BA=\left[\begin{matrix} \frac{\sqrt3}{2}&\frac{1}{2} \\  \frac{1}{2}&-\frac{\sqrt3}{2} \end{matrix}\right]$
  \end{enumerate}
}
\item
 Continuation of (7): What is the standard matrix for $U\colon\R^2\to\R^2$ which first reflects
 points at the line with equation $y=x$ and then rotates points around the origin by $60^\circ$ counter clockwise?
 Compare $T$ and $U$.
 \sol{
  $AB=\left[\begin{matrix} -\frac{\sqrt3}{2}&\frac{1}{2} \\  \frac{1}{2}&\frac{\sqrt3}{2} \end{matrix}\right]$
  The matrices are the same exept for the signs of the top left and bottom right elements being switched.
 }
\end{enumerate}

% \begin{thebibliography}{1}
% \bibitem{La-LA}
%  David C. Lay, Steven R. Lay, and Judi J. McDonald.
% \newblock Linear Algebra and Its Applications.
% \newblock Addison-Wesley, 5th edition, 2015.
% \end{thebibliography}
\end{document}




