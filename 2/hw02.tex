\documentclass[12pt,a4paper]{amsart}
\usepackage{geometry}
\geometry{left = 30mm, right = 30mm}
\addtolength{\textheight}{+25mm}
\thispagestyle{empty}
\usepackage{versions}
\newif\ifsol\solfalse

%\soltrue % comment to hide solution

\ifsol \newenvironment{solution}{\par\noindent\textbf{Solution:\\}}{\hfill\qed\\} 
\else \excludeversion{solution} \fi
\ifsol \newenvironment{grading}{\par\noindent\textbf{Grading:}}{\hfill\\}
\else \excludeversion{grading} \fi

\usepackage{amsmath}
\usepackage{amssymb}
\usepackage[lmargin=71pt, tmargin=1.2in]{geometry}

\newcommand{\ba}{\mathbf{a}}
\newcommand{\bb}{\mathbf{b}}
\newcommand{\bo}{\mathbf{0}}
\newcommand{\bp}{\mathbf{p}}
\newcommand{\bq}{\mathbf{q}}
\newcommand{\bu}{\mathbf{u}}
\newcommand{\bv}{\mathbf{v}}
\newcommand{\bw}{\mathbf{w}}
\newcommand{\bx}{\mathbf{x}}
\newcommand{\by}{\mathbf{y}}
\newcommand{\R}{\mathbb{R}}
\newcommand{\Z}{\mathbb{Z}}



\newcommand{\st}{\ |\ }
\newcommand{\Nul}{\mathrm{Null\,}}
\newcommand{\Span}{\mathrm{Span}}

\newcommand{\ve}[2]{\left[\begin{smallmatrix} #1 \\ #2 \end{smallmatrix}\right]}
\renewcommand{\vec}[3]{\left[\begin{smallmatrix} #1 \\ #2 \\ #3 \end{smallmatrix}\right]}



\begin{document}
\begin{center}
\begin{Large}
{\bf Math 2135 - Assignment 2}
\printanswers
\end{Large}

\bigskip
Due September 14, 2024
\end{center}

\thispagestyle{empty}

\bigskip


% Solve all systems of linear equations by row reduction (Gaussian elimination) and give the solutions in
% parametric vector form.


\begin{enumerate}

\item
 Is $\bb$ a linear combination of the vectors $\ba_1,\ba_2$?
 \[ \ba_1 = \left[ \begin{array}{c} 1 \\ 2 \\ 1 
\end{array} \right],
\ba_2 = \left[ \begin{array}{c} -2 \\ -3 \\ 3 
\end{array} \right],
%\ba_3 = \left[ \begin{array}{c} -6 \\ 7 \\ 5 
%\end{array} \right],
\bb = \left[ \begin{array}{c} 1 \\ -2 \\ 3 
\end{array} \right] \]

\begin{solution}
  HI
\end{solution}




\item
 Is $\mathbf{b}\in \mathrm{Span}\{\mathbf{a}_1,\mathbf{a}_2,\mathbf{a}_3\}$ for
 \[ \mathbf{a}_1 = \left[ \begin{array}{c} 1 \\ 0 \\ 1 
\end{array} \right],
\mathbf{a}_2 = \left[ \begin{array}{c} -2 \\ 3 \\ -2 
\end{array} \right],
\mathbf{a}_3 = \left[ \begin{array}{c} -6 \\ 7 \\ 5 
\end{array} \right],
\mathbf{b} = \left[ \begin{array}{c} 11 \\ -5 \\ 9 
\end{array} \right]? \]


\item
 For which values of $a$ is $\bb$ in the plane spanned by $\bv_1$ and $\bv_2$?
 \[ \bv_1 = \left[ \begin{array}{c} 1 \\ 0 \\ -2 
\end{array} \right],
\bv_2 = \left[ \begin{array}{c} -2 \\ 1 \\ 7 
\end{array} \right],
\bb = \left[ \begin{array}{c} a \\ -3 \\ -5 
\end{array} \right] \]



\item
%\begin{enumerate}
%\item
 Find vectors $\bv_1,\dots,\bv_k\in\R^3$ that span the plane in $\R^3$ with equation $x-2y+3z = 0$.
 How many do you need?

 Hint: Write down a parametrized solution for the equation. 


\item
 Are the following true or false? Explain your answers.
\begin{enumerate}
\item
 For every $A\in\R^{2\times 3}$ with $2$ pivots, $Ax=0$ has a nontrivial solution.
  
\item
 For every $A\in\R^{2\times 3}$ with $2$ pivots and every $\bb\in\R^2$, $Ax=\bb$ is consistent.

\item
 The vector $3\bv_1$ is a linear combination of the vectors $\bv_1,\bv_2$.

\item
 For $\bv_1,\bv_2\in\mathbb{R}^3$, Span$(\bv_1,\bv_2)$ is always a plane 
 through the origin.
\end{enumerate}



\item \cite[cf. Section 1.5, Ex 17]{La-LA}
Let 
\[ 
A=\left[\begin{matrix} 2 & 2 & 4 \\ -4 & -4 & -8 \\ 0 & -3 & -3 
\end{matrix}\right], \quad 
\bb=\left[\begin{matrix} 8 \\ -16 \\ 12 \end{matrix}\right], \quad
\mathbf{0}=\left[\begin{matrix} 0 \\ 0 \\ 0 \end{matrix}\right]. 
\]
Solve the equations $A\bx=\bb$ and  $A\bx=\mathbf{0}$. Express both solution sets in parametric vector form.
Give a geometric description of the solution sets.

\item\label{mat12}
\begin{enumerate}
\item Which of the vectors $\bu,\bv,\bw$ are in the nullspace of $A$, $\Nul A$?
\[ 
\bu = \left[\begin{matrix} 0 \\ 0 \\ 0 \\ 0 \end{matrix}\right], \quad
\bv = \left[\begin{matrix} -2 \\ 0 \\ 4 \\ -2 \end{matrix}\right], \quad
\bw = \left[\begin{matrix} 1 \\ 1 \\ -2 \\ 1 \end{matrix}\right], \quad
A = \left[ \begin{matrix}
 0 & 0 & 2 & 4 \\
 2 & -4 & 1 & 0 \\
 -3 & 6 & 2 & 7
 \end{matrix} \right] \]

\item  Solve $A\bx=\bo$ and give the solution in parametric vector form.

\item  Find vectors $v_1,\dots,v_k \in \R^4$ such that $\Nul A = \Span\{ v_1,\dots,v_k \}$.
\end{enumerate}




\item
Show the following:

\bigskip

\noindent {\bf Theorem.} Suppose $A\bx=\bb$ has a solution $\mathbf{p}$. Then the set of all solutions of $A\bx=\bb$ is
\[ \mathbf{p}+\Nul A = \{\mathbf{p}+\mathbf{v}\ |\ \mathbf{v} \in \Nul A\}. \]

\bigskip

\noindent Hint: For the proof suppose $A\bx=\bb$ has a solution $\bp$ and use 2 steps:
\begin{enumerate}
\item Show that if $\bv$ is in $\Nul A$, then $\bp+\bv$ is also a solution for $A\bx=\bb$.
\item Show that if $\bq$ is a solution for $A\bx=\bb$, then $\mathbf{q}-\bp$ is in $\Nul A$.
\end{enumerate}
 
\end{enumerate}




\begin{thebibliography}{1}
\bibitem{La-LA}
 David C. Lay, Steven R. Lay, and Judi J. McDonald.
\newblock Linear Algebra and Its Applications.
\newblock Addison-Wesley, 5th edition, 2015.
\end{thebibliography}




\end{document}







%%% Local Variables: 
%%% mode: latex
%%% TeX-master: t
%%% End: 
