\documentclass[12pt,a4paper]{exam}
%\usepackage{geometry}
%\geometry{left = 30mm, right = 30mm}
\addtolength{\textheight}{+25mm}
\thispagestyle{empty}
\usepackage{versions}
%\newif\ifsol\solfalse

%\soltrue % comment to hide solution

%\ifsol \newenvironment{solution}{\par\noindent\textbf{Solution:\\}}{\hfill\qed\\} 
%\else \excludeversion{solution} \fi
%\ifsol \newenvironment{grading}{\par\noindent\textbf{Grading:}}{\hfill\\}
%\else \excludeversion{grading} \fi

\usepackage{amsmath}
\usepackage{amssymb}
\usepackage[lmargin=71pt, tmargin=1.2in]{geometry}

\newcommand{\ba}{\mathbf{a}}
\newcommand{\bb}{\mathbf{b}}
\newcommand{\bo}{\mathbf{0}}
\newcommand{\bp}{\mathbf{p}}
\newcommand{\bq}{\mathbf{q}}
\newcommand{\bu}{\mathbf{u}}
\newcommand{\bv}{\mathbf{v}}
\newcommand{\bw}{\mathbf{w}}
\newcommand{\bx}{\mathbf{x}}
\newcommand{\by}{\mathbf{y}}
\newcommand{\R}{\mathbb{R}}
\newcommand{\Z}{\mathbb{Z}}



\newcommand{\st}{\ |\ }
\newcommand{\Nul}{\mathrm{Null\,}}
\newcommand{\Span}{\mathrm{Span}}

\newcommand{\ve}[2]{\left[\begin{smallmatrix} #1 \\ #2 \end{smallmatrix}\right]}
\renewcommand{\vec}[3]{\left[\begin{smallmatrix} #1 \\ #2 \\ #3 \end{smallmatrix}\right]}



\begin{document}
\begin{center}
\begin{Large}
{\bf Math 2135 - Assignment 2}
\end{Large}

\bigskip
Due September 14, 2024
\end{center}

\thispagestyle{empty}

\bigskip




\printanswers
\renewcommand{\solutiontitle}{\noindent\textbf{Ans:}\enspace}
% Solve all systems of linear equations by row reduction (Gaussian elimination) and give the solutions in
% parametric vector form.


\begin{enumerate}

\item
 Is $\bb$ a linear combination of the vectors $\ba_1,\ba_2$?
 \[ \ba_1 = \left[ \begin{array}{c} 1 \\ 2 \\ 1 
\end{array} \right],
\ba_2 = \left[ \begin{array}{c} -2 \\ -3 \\ 3 
\end{array} \right],
%\ba_3 = \left[ \begin{array}{c} -6 \\ 7 \\ 5 
%\end{array} \right],
\bb = \left[ \begin{array}{c} 1 \\ -2 \\ 3 
\end{array} \right] \]

\begin{solution}
  \begin{itemize}
    \item Create matrix to represent system:\\
      \[
      \left[ \begin{array}{ccc}
          1 & -2 & 1 \\ 2 & -3 & -2 \\ 1 & 3 & 3
      \end{array} \right]
      \]
    \item Row reduce:\\
      \[
      \left[ \begin{array}{ccc}
          1 & -2 & 1 \\ 0 & 1 & -4 \\ 0 & 0 & 22
      \end{array} \right]
      \]
    \item $\bb$ is not a linear combination of the vectors because the the system represented by the augmented matrix is inconsistent.
  \end{itemize}
\end{solution}




\item
 Is $\mathbf{b}\in \mathrm{Span}\{\mathbf{a}_1,\mathbf{a}_2,\mathbf{a}_3\}$ for
 \[ \mathbf{a}_1 = \left[ \begin{array}{c} 1 \\ 0 \\ 1 
\end{array} \right],
\mathbf{a}_2 = \left[ \begin{array}{c} -2 \\ 3 \\ -2 
\end{array} \right],
\mathbf{a}_3 = \left[ \begin{array}{c} -6 \\ 7 \\ 5 
\end{array} \right],
\mathbf{b} = \left[ \begin{array}{c} 11 \\ -5 \\ 9 
\end{array} \right]? \]

\begin{solution}
  \begin{itemize}
    \item Create matrix to represent system:\\
      \[
      \left[ \begin{array}{cccc}
          1 & -2 & -6 & 11 \\ 0 & 3 & 7 & -5 \\ 1 & -2 & 5 & 9
      \end{array} \right]
      \]
    \item Row reduce:\\
      \[
      \left[ \begin{array}{cccc}
          1 & -2 & -6 & 11 \\ 0 & 3 & 7 & -5 \\ 0 & 0 & 11 & -2
      \end{array} \right]
      \]
    \item $\bb$ is a linear combination of the vectors because the the system represented by the augmented matrix is consistent.
  \end{itemize}
\end{solution}



\item
 For which values of $a$ is $\bb$ in the plane spanned by $\bv_1$ and $\bv_2$?
 \[ \bv_1 = \left[ \begin{array}{c} 1 \\ 0 \\ -2 
\end{array} \right],
\bv_2 = \left[ \begin{array}{c} -2 \\ 1 \\ 7 
\end{array} \right],
\bb = \left[ \begin{array}{c} a \\ -3 \\ -5 
\end{array} \right] \]


\begin{solution}
  \begin{itemize}
    \item Create matrix to represent system:\\
      \[
      \left[ \begin{array}{ccc}
          1 & -2 & a \\ 0 & 1 & -3 \\ -2 & 7 & -5
      \end{array} \right]
      \]
    \item Row reduce:\\
      \[
      \left[ \begin{array}{ccc}
        1 & -2 & a \\ 0 & 1 & -3 \\ 0 & 0 & 4+2a
      \end{array} \right]
      \]
    \item The system must be consistent for $\bb$ to be in the plane.
          The last row represents the equation $0=4+2a$, so if we solve this, we find the value of $a$ that allows the system to be consistent.
          When the equation is solved we get $a=-2$ as the solution.
  \end{itemize}
\end{solution}




\item
%\begin{enumerate}
%\item
 Find vectors $\bv_1,\dots,\bv_k\in\R^3$ that span the plane in $\R^3$ with equation $x-2y+3z = 0$.
 How many do you need?

 Hint: Write down a parametrized solution for the equation. 

\begin{solution}
  \begin{itemize}
    \item We can see from the equation that the plane goes through $(0,0,0)$. We can choose two arbitrary non-colinear points and the vectors to those points will span the plane.
    \item Setting $x=0$ and $y=3$, we get $z=2$ and the point $(0,3,2)$. Setting $y=0$ and $x=3$ we get $z=-1$ and the point $(3,0,-1)$.
    \item The vectors from the origin to these point are $ \left[ \begin{array}{c} 0 \\ 3 \\ 2 \end{array} \right]$ and $\left[ \begin{array}{c} 3 \\ 0 \\ -1 \end{array} \right]$.
    \item Since these vectors both lie on the plane and are not colinear, they span the plane.
  \end{itemize}
\end{solution}


\item
 Are the following true or false? Explain your answers.
\begin{enumerate}
\item
 For every $A\in\R^{2\times 3}$ with $2$ pivots, $Ax=0$ has a nontrivial solution.
  
\item
 For every $A\in\R^{2\times 3}$ with $2$ pivots and every $\bb\in\R^2$, $Ax=\bb$ is consistent.

\item
 The vector $3\bv_1$ is a linear combination of the vectors $\bv_1,\bv_2$.

\item
 For $\bv_1,\bv_2\in\mathbb{R}^3$, Span$(\bv_1,\bv_2)$ is always a plane 
 through the origin.
\end{enumerate}


\begin{solution}
  \begin{enumerate}
    \item
    \item
    \item True, $3\bv_1+0\bv_2=3\bv_1$
    \item False, if the vectors are colinear then their span will only be a line.
  \end{enumerate}
\end{solution}


\item \cite[cf. Section 1.5, Ex 17]{La-LA}
Let 
\[ 
A=\left[\begin{matrix} 2 & 2 & 4 \\ -4 & -4 & -8 \\ 0 & -3 & -3 
\end{matrix}\right], \quad 
\bb=\left[\begin{matrix} 8 \\ -16 \\ 12 \end{matrix}\right], \quad
\mathbf{0}=\left[\begin{matrix} 0 \\ 0 \\ 0 \end{matrix}\right]. 
\]
Solve the equations $A\bx=\bb$ and  $A\bx=\mathbf{0}$. Express both solution sets in parametric vector form.
Give a geometric description of the solution sets.

\begin{solution}
  \begin{itemize}
    \item Augmented Matrix:\\
      \[
        \left[\begin{array}{cccc} 2 & 2 & 4 & 8 \\ -4 & -4 & -8 & -16 \\ 0 & -3 & -3 & 12
        \end{array}\right]
      \]
    \item Row reduce:\\
      \[
        \left[\begin{array}{cccc} 1 & -1 & 0 & 12 \\ 0 & 1 & 1 & -4 \\ 0 & 0 & 0 & 0
        \end{array}\right]
      \]
  \end{itemize}
  \begin{enumerate}
    \item Solve for $Ax=b$:\\
      Equation 2: $x_2=-x_3-4$ 
      Equation 1: $x_1=12+x_2=8-x_3$
      if we set $t=x_3$ we get the following parametric equation: 
      $x=\left[\begin{array}{c} 8 \\ -4 \\ 0 \end{array}\right] + t\left[\begin{array}{c} -1 \\ -1 \\ 1 \end{array}\right]$\\
      The system of equations represents a line that is the intersection of 2 planes. The line goes through the point $(8,-4,0)$.
    \item Solve for $Ax=0$;\\
      The row reduced array is the same exept the rightmost column is all zeroes.\\
      Equation 2: $x_2=-x_3$ 
      Equation 1: $x_1=x_2=-x_3$


      if we set $t=x_3$ we get the following parametric equation: 
      $x=t\left[\begin{array}{c} -1 \\ -1 \\ 1 \end{array}\right]$\\
      The system of equations represents a line that is the intersection of 2 planes. The line goes through the origin.

  \end{enumerate}
\end{solution}



\item\label{mat12}
\begin{enumerate}
\item Which of the vectors $\bu,\bv,\bw$ are in the nullspace of $A$, $\Nul A$?
\[ 
\bu = \left[\begin{matrix} 0 \\ 0 \\ 0 \\ 0 \end{matrix}\right], \quad
\bv = \left[\begin{matrix} -2 \\ 0 \\ 4 \\ -2 \end{matrix}\right], \quad
\bw = \left[\begin{matrix} 1 \\ 1 \\ -2 \\ 1 \end{matrix}\right], \quad
A = \left[ \begin{matrix}
 0 & 0 & 2 & 4 \\
 2 & -4 & 1 & 0 \\
 -3 & 6 & 2 & 7
 \end{matrix} \right] \]

\item  Solve $A\bx=\bo$ and give the solution in parametric vector form.

\item  Find vectors $v_1,\dots,v_k \in \R^4$ such that $\Nul A = \Span\{ v_1,\dots,v_k \}$.
\end{enumerate}


\begin{solution}
  \begin{enumerate}
    \item
      \begin{enumerate}
        \item \[ \left[\begin{matrix} 0 \\ 0 \\ 0 \\ 0 \end{matrix}\right] \] is in the nullspace because it is clear that if we multiply $A$ by all zeroes we will get the zero vector.
        \item \[ \left[\begin{matrix} -2 \\ 0 \\ 4 \\ -2 \end{matrix}\right] \cdot \left[\begin{matrix}
          0 & 0 & 2 & 4 \\
          2 & -4 & 1 & 0 \\
          -3 & 6 & 2 & 7
          \end{matrix}\right]
          =
          \left[\begin{matrix} 0+0+8-8 \\ -4+0+4+0 \\ 6+0+8-14 \end{matrix}\right]
          =
          \left[\begin{matrix} 0 \\ 0 \\ 0 \end{matrix}\right]
          \] so the vector is in the nullspace of $A$.
        \item \[ \left[\begin{matrix} 1 \\ 1 \\ -2 \\ 1 \end{matrix}\right] \cdot \left[\begin{matrix}
          0 & 0 & 2 & 4 \\
          2 & -4 & 1 & 0 \\
          -3 & 6 & 2 & 7
          \end{matrix}\right]
          =
          \left[\begin{matrix} 0+0-4+4 \\ 2-4-2+0 \\ -3+6-4+7 \end{matrix}\right]
          =
          \left[\begin{matrix} 0 \\ -4 \\ 6 \end{matrix}\right]
          \] so the vector is not in the nullspace of $A$.
      \end{enumerate}
    \item
    \begin{itemize}
      \item Augmented matrix:\\
        \[ \left[ \begin{matrix}
            0 & 0 & 2 & 4 & 0\\
            2 & -4 & 1 & 0 & 0\\
            -3 & 6 & 2 & 7 & 0
         \end{matrix} \right] \]
      \item Row reduce:\\
        \[ \left[ \begin{matrix}
            1 & -2 & 0 & -1 & 0\\
            0 & 0 & 1 & 2 & 0\\
            0 & 0 & 0 & 0 & 0
         \end{matrix} \right] \]
      \item Equations:\\
        Equation 2: $x_3=-2x_4$. 
        Equation 1: $x_1=2x_2+x_4$. 
        $x_2$ and $x_4$ are both free so we set $t=x_2$ and $u=x_4$.
      \item Parametric equation:\\
        $x=t\left[\begin{array}{c} 2 \\ 1 \\ 0 \\ 0 \end{array}\right] + u\left[\begin{array}{c} 1 \\ 0 \\ -2 \\ 1 \end{array}\right]$\\
    \end{itemize}
    \item The span of a system is just the set of vectors for which all possible linear combination of them is
          equivilant to the system's solutions. Therefore we can just use the vector from the previous question,
          so $\Nul A = \Span \{\left[\begin{array}{c} 2 \\ 1 \\ 0 \\ 0 \end{array}\right], \left[\begin{array}{c} 1 \\ 0 \\ -2 \\ 1 \end{array}\right]\}$
  \end{enumerate}
\end{solution}



\item
Show the following:

\bigskip

\noindent {\bf Theorem.} Suppose $A\bx=\bb$ has a solution $\mathbf{p}$. Then the set of all solutions of $A\bx=\bb$ is
\[ \mathbf{p}+\Nul A = \{\mathbf{p}+\mathbf{v}\ |\ \mathbf{v} \in \Nul A\}. \]

\bigskip

\noindent Hint: For the proof suppose $A\bx=\bb$ has a solution $\bp$ and use 2 steps:
\begin{enumerate}
\item Show that if $\bv$ is in $\Nul A$, then $\bp+\bv$ is also a solution for $A\bx=\bb$.
\item Show that if $\bq$ is a solution for $A\bx=\bb$, then $\mathbf{q}-\bp$ is in $\Nul A$.
\end{enumerate}
\begin{solution}
  \begin{itemize}
    \item First, suppose that $A\bx=b$ has some solution $\bx=\bp$.
    \item If we then take any $\bv\in\Nul A$ 
  \end{itemize}
\end{solution}
\end{enumerate}




\begin{thebibliography}{1}
\bibitem{La-LA}
 David C. Lay, Steven R. Lay, and Judi J. McDonald.
\newblock Linear Algebra and Its Applications.
\newblock Addison-Wesley, 5th edition, 2015.
\end{thebibliography}




\end{document}







%%% Local Variables: 
%%% mode: latex
%%% TeX-master: t
%%% End: 
